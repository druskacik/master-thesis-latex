Cieľom práce bolo vytvoriť modely logistickej regresie určené na predikciu bankrotu v slovenskom prostredí.
Na voľbu vhodných kombinácií parametrov sme použili dve metódy – \emph{lasso} a \emph{BACE} (\emph{Bayesian averaging of classical estimates}).
Výsledky naznačujú, že naše modely majú v slovenskom prostredí lepšiu predikčnú schopnosť ako zaužívané modely Altmanovo Z-skóre a index IN05.

Pri vytváraní modelov sme kládli vysoký dôraz na jednoduchosť a ľahkú interpretovateľnosť výsledných vzorcov.
Ukazuje sa, že metóda \emph{BACE}, ktorá medzi štatistikmi pravdepodobne nie je taká známa ako iné metódy voľby modelu,
môže byť vhodnou metódou práve pri problémoch, kde výskumník dbá na interpretovateľnosť a pochopiteľnosť výsledkov.
Výstupom metódy sú okrem iného aj aposteriórne pravdepodobnosti zahrnutia jednotlivých vysvetľujúcich premenných do „správneho“ modelu,
a práve tieto hodnoty môžu mať pre skúmanú oblasť vysokú výpovednú hodnotu.

V ďalšom výskume by sme sa mohli zamerať na rigoróznejšiu analýzu metódy \emph{BACE}, napr. porovnanie dvoch prístupov k jej implementácii
(kompletná enumerácia vopred danej časti modelov, resp. iteračná metóda náhodne generujúca nové a nové modely).
Výsledky našej práce naznačujú, že ich výstupy sú veľmi podobné.
Zaujímavou témou ďalšieho výskumu môže byť porovnanie dvoch implementácií metódy \emph{BACE} v rôznych simulovaných prostrediach.

Modely vytvorené v našej práci nadobúdajú jej zverejnením verejný charakter a môžu byť voľne využité inštitúciami i
jednotlivcami na ohodnotenie kreditného rizika slovenských firiem.
Aj z tohto dôvodu sme počas celého výskumu dávali veľký dôraz na to, aby procesy vytvorenia modelov boli čo najtransparentnejšie
a aby výsledné vzorce mohli byť ľahko interpretovateľné.
Máme za to, že v niektorých situáciách je vhodnejšie nesústrediť sa len na predikčnú schopnosť matematických modelov,
ktoré vytvárame, ale dbať aj na to, aby každý jeden výstup nášho modelu bol obhájiteľný.
To často neplatí pri využití niektorých modernejších štatistických metód, ktorých výstupom býva model tzv. „čiernej skrinky“,
ktorý síce môže mať vyššiu predikčnú schopnosť, ale jeho interné fungovanie nemusí byť navonok jasné.
Uvedomme si, že sme to my, matematici, ktorí v konečnom dôsledku nesieme zodpovednosť za výstupy našich modelov.
Nech aj to je mementom pre výskumníkov, ktorí si vezmú za cieľ vytvorenie verejných matematických modelov.