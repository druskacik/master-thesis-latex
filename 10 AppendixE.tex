V tejto prílohe sa bližšie venujeme výpočtovým aspektom iteračnej metódy BACE opísanej v časti\autoref{model_bace_2}.
Pripomeňme si, že pri celkovom počte vysvetľujúcich \(K\) a očakávanom počte premenných v „správnom“ modeli \(\bar{k}\)
sme za apriórne pravdepodobnosti zahrnutia premenných zvolili hodnotu \(\bar{k}/K\), a apriórne pravdepodobnosti modelov sme dopočítali ako:

\[
p_{\text{prior}}(M_i) = \left( \frac{\bar{k}}{K} \right)^{k_i} \left( 1 - \frac{\bar{k}}{K} \right)^{K - k_i},
\]

kde \( k_i \) je počet premenných zahrnutých do modelu \( M_i \).
Po enumerácii nejakého počtu modelov \(c\) spočítavame odhad strednej hodnoty parametra \(\beta\) a aposteriórnu pravdepodobnosť zahrnutia vysvetľujúcej premennej \(X_j\) pomocou vzťahov:

\[
    E(\beta | y) = \sum_{i = 1}^{c} p(M_i | y) \hat{\beta}^{(i)},
\]

\[
    p(\beta_j \neq 0 | y) = \sum_{i = 1}^{c} p(M_i | y) I_{\beta_{j, i} \neq 0},
\]

kde \(\hat{\beta}^{(i)}\) je odhad parametra \(\beta\) pre model \(M_i\), a \(p(M_i | y)\) je aposteriórna pravdepodobnosť modelu \(M_i\), ktorú aproximujeme vzťahom:

\[
    p(M_i | y) = \frac{p(M_i) e^{-\frac{1}{2}BIC(M_i)}}{\sum_{i = 1}^{K} p(M_i) e^{-\frac{1}{2}BIC(M_i)}}.
\]

Pri prístupe, v ktorom sme enumerovali všetky modely s \(\bar{k} = 5\) vysvetľujúcimi premennými, je postup priamočiary –
spočítať odhady \(\beta\) a hodnoty \emph{BIC} pre všetky modely a dopočítať strednú hodnotu \(\beta\) a aposteriórne pravdepodobnosti zahrnutia premenných.
V prípade iteračnej metódy, pri ktorej generujeme nové náhodné modely, až kým hodnota odhadu beta neskonverguje,
treba odhad beta počítať veľakrát počas behu algoritmu, čo robí problém výpočtovo oveľa náročnejším.

Pri implementácii algoritmu sme využili skutočnosť, že odhad beta spočítaný po enumerácii \(I\) modelov vieme využiť pri výpočte odhadu beta po enumerácii \(I + J\) modelov.
Toto tvrdenie ukážeme pre výpočet odhadu \(\beta\) v iteráciách \(I\) a \(I + 1\).
Označme \(\hat{\beta}^{(1:I)}\) odhad parametra \(\beta\) v iterácii \(I\) daný vzťahom:

% \begin{gather*}
%     \hat{\beta}^{(I)} = \sum_{i = 1}^{I} p(M_i | y) \hat{\beta} = \sum_{i = 1}^{I} \frac{p(M_i) e^{-\frac{1}{2}BIC(M_i)}}{\sum_{i = 1}^{K} p(M_i) e^{-\frac{1}{2}BIC(M_i)}} \hat{\beta} = \\
%     \frac{1}{\sum_{i = 1}^{K} p(M_i) e^{-\frac{1}{2}BIC(M_i)}} \sum_{i = 1}^{I} p(M_i) e^{-\frac{1}{2}BIC(M_i)} \hat{\beta}
% \end{gather*}

\[
    \hat{\beta}^{(1:I)} = \sum_{i = 1}^{I} p(M_i | y) \hat{\beta}^{(i)} = \sum_{i = 1}^{I} \frac{p(M_i) e^{-\frac{1}{2}BIC(M_i)}}{\sum_{j = 1}^{I} p(M_j) e^{-\frac{1}{2}BIC(M_j)}} \hat{\beta}^{(i)} =
\]
\begin{equation} \label{appendix__expected_value}
    = \frac{\sum_{i = 1}^{I} p(M_i) e^{-\frac{1}{2}BIC(M_i)} \hat{\beta}^{(i)}}{\sum_{i = 1}^{I} p(M_i) e^{-\frac{1}{2}BIC(M_i)}} = \frac{1}{d_I} \sum_{i = 1}^{I} p(M_i) e^{-\frac{1}{2}BIC(M_i)} \hat{\beta}^{(i)},
\end{equation}

pričom sme zaviedli značenie \(d_n := \sum_{i = 1}^{n} p(M_i) e^{-\frac{1}{2}BIC(M_i)}\).
Pre odhad strednej hodnoty \(\beta\) v iterácii \(I + 1\) (značíme \(\hat{\beta}^{(1:I + 1)}\)) potom platí:

\[
    \hat{\beta}^{(1:I + 1)} = \frac{1}{\sum_{i = 1}^{I + 1} p(M_i) e^{-\frac{1}{2}BIC(M_i)}} \sum_{i = 1}^{I + 1} p(M_i) e^{-\frac{1}{2}BIC(M_i)} \hat{\beta}^{(i)} = 
\]
\[
    = \frac{1}{d_I + p(M_{I + 1}) e^{-\frac{1}{2}BIC(M_{I + 1})}} \left( \left( \sum_{i = 1}^{I} p(M_i) e^{-\frac{1}{2}BIC(M_i)} \hat{\beta}^{(i)} \right) + p(M_{I + 1}) e^{-\frac{1}{2}BIC(M_{I + 1})} \hat{\beta}^{(I+1)} \right) = 
\]
\[
    = \frac{d_I}{d_I + p(M_{I + 1}) e^{-\frac{1}{2}BIC(M_{I + 1})}} \hat{\beta}^{(1:I)} + \frac{p(M_{I + 1}) e^{-\frac{1}{2}BIC(M_{I + 1})} \hat{\beta}^{(I+1)}}{d_I + p(M_{I + 1}) e^{-\frac{1}{2}BIC(M_{I + 1})}}.
\]

Ako vidíme, v iterácii \(I + 1\) nám na výpočet odhadu \(\beta\) stačí odhad \(\hat{\beta}^{(1:I)}\) z predošlej iterácie a hodnota \(d_I\),
ktorá predstavuje súčet apriórnych pravdepodobností modelov prevážených hodnotami \(e^{-\frac{1}{2}BIC(M_{.})}\).
V praxi to znamená, že po výpočte odhadu \(\beta\) v iterácii I môže algoritmus „zahodiť“ apriórne aj aposteriórne pravdepodobnosti enumerovaných modelov,
odhady parametra \(\beta\) pre tieto modely a aj ich hodnoty \emph{BIC}.
Celá informácia potrebná pre ďalší beh algoritmu sa dá zhrnúť len do dvoch hodnôt:
samotného odhadu \(\hat{\beta}^{(1:I)}\) strednej hodnoty parametra \(\beta\) po iterácii \(I\) a hodnoty \(d_I := \sum_{i = 1}^{I} p(M_i) e^{-\frac{1}{2}BIC(M_i)}\).
Napriek tomu, že odhad strednej hodnoty \(\beta\) definujeme pomocou sumy cez všetky enumerované modely (viď vzorec (\ref{posterior_expected_value})),
pri optimálnej implementácii sumu cez všetky modely nebudeme musieť počítať vôbec.

Z uvedeného vyplýva, že algoritmus využívajúci vyššie uvedené vzťahy bude fungovať konštantne rýchlo aj pre obrovský počet enumerovaných modelov (potenciálne desiatky až stovky miliónov).

Podobnú logiku môžeme aplikovať aj na výpočet aposteriórnych pravdepodobností zahrnutia jednotlivých vysvetľujúcich premenných.
Ak pre aposteriórnu pravdepobnosť zahrnutia premennej \(X_j\) do modelu po iterácii \(I\) platí:

\[
    p(\beta_j \neq 0 | y)^{(1:I)} = \sum_{i = 1}^{I} p(M_i | y) I_{\beta_{j, i} \neq 0} = \frac{1}{\sum_{i = 1}^{I} p(M_i) e^{-\frac{1}{2}BIC(M_i)}} \sum_{i = 1}^{I} p(M_i) e^{-\frac{1}{2}BIC(M_i)} I_{\beta_{j, i} \neq 0},
\]

tak po iterácii I + 1 platí:

\[
    p(\beta_j \neq 0 | y)^{(1:I + 1)} =
\]
\[
    = \frac{d_I}{d_I + p(M_{I + 1}) e^{-\frac{1}{2}BIC(M_{I + 1})}} p(\beta_j \neq 0 | y)^{(1:I)} + \frac{p(M_{I + 1}) e^{-\frac{1}{2}BIC(M_{I + 1})} I_{\beta_{j, I + 1} \neq 0}}{d_I + p(M_{I + 1}) e^{-\frac{1}{2}BIC(M_{I + 1})}}.
\]

Opäť vidíme, že algoritmus si v iterácii \(I\) potrebuje zapamätať len dve hodnoty: \(d_I\) a \( p(\beta_j \neq 0 | y)^{(1:I)} \)
(odhad aposteriórnej pravdepodobnosti zahrnutia premennej po \(I\)-tej iterácii).

V praxi nemá význam počítať odhad strednej hodnoty \(\beta\) a aposteriórne pravdepodobnosti zahrnutia premenných v každej iterácii, ale až po enumerácii väčšieho množstva modelov.
V tejto práci sme enumerovali odhad \(\beta\) každých \(10000\) iterácií.
Aj v takomto prípade ale nemusíme zakaždým pri výpočte robiť sumu cez všetky modely.
Ak v iterácii I je odhad \(\beta\) daný vzťahom (\ref{appendix__expected_value}), tak v iterácii \(I + 10000\) môžeme odhad \(\beta\) spočítať ako:

\[
    \hat{\beta}^{(1:I + 10000)} =
\]
\[
    = \frac{d_I}{d_I + \sum_{i = I + 1}^{I + 10000} p(M_i) e^{-\frac{1}{2}BIC(M_i)}} \hat{\beta}^{(1:I)} + \frac{1}{d_I + \sum_{i = I + 1}^{I + 10000} p(M_i) e^{-\frac{1}{2}BIC(M_i)}} \sum_{i = I + 1}^{I + 10000} p(M_i) e^{-\frac{1}{2}BIC(M_i)} \hat{\beta}^{(i)}.
\]