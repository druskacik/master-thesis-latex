\documentclass[a4paper,12pt,twoside]{article}

%PACKAGES

\usepackage[T1]{fontenc}
\usepackage{lmodern}

\usepackage[utf8]{inputenc}
\usepackage[slovak]{babel}
\usepackage{comment}

\usepackage{array}

\usepackage{graphicx,amsmath,amssymb, amsthm, multicol}
\usepackage{pdfpages}
\usepackage{float}

\usepackage[nottoc]{tocbibind}
\usepackage{mathrsfs}
\usepackage{psfrag}
\usepackage[small,bf]{caption}
\usepackage{ifthen}
\usepackage{titlesec}
\usepackage{makecell}

\usepackage{xcolor}
\usepackage{listings}
\usepackage{enumitem}
\usepackage{longtable}
\usepackage{lastpage}

\usepackage[toc,page]{appendix}

\usepackage{url}
\urlstyle{rm}
\renewcommand\UrlFont{\color{black}}

\usepackage{pgfplots}
\pgfplotsset{compat=1.16}

\usepackage{tikz}
\usetikzlibrary{positioning,chains,fit,shapes,calc}

% \renewcommand{\familydefault}{\rmdefault}

\newtheorem{defin}{Definícia}[section]
\newtheorem{theorem}[defin]{Veta}
\newtheorem{prop}[defin]{Tvrdenie}
\newtheorem{lema}[defin]{Lema}
\newtheorem{cor}[defin]{Dôsledok}
\newtheorem{hypoteza}{Hypotéza}[section]
\newtheoremstyle{comment}{}{}{}{}{}{:}{ }{#1}
\theoremstyle{comment}
\newtheorem{dokaz}{Dôkaz}[section]
\newtheorem{com}{\textit{Poznámka}}

\usepackage{hyperref}                                     
\hypersetup{%  http://www.tug.org/applications/hyperref/
bookmarksnumbered,
pdfstartview={FitH},
linkcolor=black,
citecolor=black,
colorlinks=true,
pdfpagemode={None},
plainpages=false
}%

%%\usepackage{fullpage}
%%\setlength{\topmargin}{-0.5cm}
%%\setlength{\headheight}{0cm}
%%\setlength{\headsep}{0in}
\setlength{\textheight}{24cm}
\setlength{\textwidth}{15.5cm}
\addtolength{\voffset}{-1.2cm}
\addtolength{\hoffset}{-0.3cm}
%%\addtolength{\rightmargin}{-1cm}
\setlength{\parindent}{0.5cm}
\setlength{\parskip}{0in}
\linespread{1.5}

\setcounter{MaxMatrixCols}{20}

\titleformat{\paragraph}
{\normalfont\normalsize\bfseries}{\theparagraph}{1em}{}
\titlespacing*{\paragraph}
{0pt}{3.25ex plus 1ex minus .2ex}{1.5ex plus .2ex}

\newcommand*{\horzbar}{\rule[.5ex]{2.5ex}{0.5pt}}
\renewcommand{\subsectionautorefname}{}
\renewcommand{\paragraphautorefname}{}

\begin{document}

  \thispagestyle{empty}
\begin{center}
{\large \bf UNIVERZITA KOMENSKÉHO V BRATISLAVE \\
FAKULTA MATEMATIKY, FYZIKY A INFORMATIKY}
\end{center}
%\begin{titlepage}
%    \rmfamily
%    \begin{center}
%      \LARGE\scshape
%      \theuniversity\\
%      \Large\upshape
%      \thefaculty\\
%      \large
%      \thedepartment
%    \end{center}
%

\vspace{2cm}
\begin{figure}[!h]
   \centering
     %\includegraphics[width=3.5cm]{logoUK.jpg}
\end{figure}

\vspace{1cm}
\begin{center}
{\large \bf Predikcia bankrotu firiem v slovenskom prostredí \\
\vspace{3cm}
DIPLOMOVÁ PRÁCA}
\end{center}

\vfill
%
\begin{multicols}{2}
{\bf
\begin{flushleft} 2022 \end{flushleft}
\begin{flushright} Bc. Róbert Druska \end{flushright} 
}
\end{multicols}


  \newpage
\thispagestyle{empty}
\begin{center}
{\large UNIVERZITA KOMENSKÉHO V BRATISLAVE \\
FAKULTA MATEMATIKY, FYZIKY A INFORMATIKY}
\end{center}


\vspace{5cm}
\begin{center}
{\large \bf Predikcia úpadku slovenských firiem \\
\vspace{3cm}
DIPLOMOVÁ PRÁCA}
\end{center}

\vfill
\begin{flushleft}
\begin{tabular}{ll}
Študijný program: & Pravdepodobnosť a matematická štatistika \\
Študijný odbor: & Aplikovaná matematika \\
Školiace pracovisko: & Katedra aplikovanej matematiky a štatistiky \\
Vedúci práce: & doc. Mgr. Radoslav Harman, PhD. \\
\end{tabular}
\end{flushleft}

\vfill
%
\begin{multicols}{2}
\begin{flushleft} Bratislava 2022 \end{flushleft}
\begin{flushright} {\bf Bc. Róbert Druska} \end{flushright}
\end{multicols}



  \includepdf[pages={1}, offset=25 -75]{zadanieRD.pdf}

  

\vglue0pt
\vfill
\thispagestyle{empty}
\paragraph{Poďakovanie}

Touto cestou by som sa chcel poďakovať svojmu školiteľovi, Radoslavovi Harmanovi, za jeho ochotu a množstvo dobrých a pravidelných rád pri písaní tejto práce.
Ďakujem tiež svojej kolegyni Monike Hrnčárovej za cenné analýzy predchádzajúce praktickým častiam práce, a svojej mame za gramatickú a štylistickú korektúru.
V neposlednom rade ďakujem svojim priateľom za oporu v posledných mesiacoch písania.


  \newpage
  \thispagestyle{empty}
\section*{Abstrakt}
DRUSKA, Róbert: Predikcia úpadku slovenských firiem [Diplomová práca],
Univerzita Komenského v Bratislave,
Fakulta matematiky, fyziky a informatiky,
Katedra aplikovanej matematiky a štatistiky,
školiteľ: doc. Mgr. Radoslav Harman, PhD.,
Bratislava, 2020, TODO s.

TODO Lorem ipsum dolor sit amet, consectetur adipiscing elit, sed do eiusmod tempor incididunt ut labore et dolore magna aliqua. Ut enim ad minim veniam, quis nostrud exercitation ullamco laboris nisi ut aliquip ex ea commodo consequat. Duis aute irure dolor in reprehenderit in voluptate velit esse cillum dolore eu fugiat nulla pariatur. Excepteur sint occaecat cupidatat non proident, sunt in culpa qui officia deserunt mollit anim id est laborum.

\begin{flushleft}
  \textbf{Kľúčové slová:} logistická regresia, TODO
\end{flushleft}

  \newpage
  \thispagestyle{empty}
\section*{Abstract}
DRUSKA, Róbert: Bankruptcy prediction [Master thesis],
Comenius University in Bratislava,
Faculty of Mathematics, Physics and Informatics,
Department of Applied Mathematics and Statistics,
supervisor: doc. Mgr. Radoslav Harman, PhD.,
Bratislava, 2020, TODO p.

TODO Lorem ipsum dolor sit amet, consectetur adipiscing elit, sed do eiusmod tempor incididunt ut labore et dolore magna aliqua. Ut enim ad minim veniam, quis nostrud exercitation ullamco laboris nisi ut aliquip ex ea commodo consequat. Duis aute irure dolor in reprehenderit in voluptate velit esse cillum dolore eu fugiat nulla pariatur. Excepteur sint occaecat cupidatat non proident, sunt in culpa qui officia deserunt mollit anim id est laborum.

\begin{flushleft}
  \textbf{Keywords:} logistic regression, TODO
\end{flushleft}
  \newpage \tableofcontents
  \setcounter{page}{7}
 % \newpage
 % \listoffigures
 % \newpage
 % \listoftables  
  \newpage

    \section*{Úvod}	          
    \markboth{ÚVOD}{ÚVOD}  
    \addcontentsline{toc}{section}{Úvod}
    
    Predikcia bankrotu firiem je predmetom akademického aj profesionálneho výskumu už po dlhé desaťročia.
Motivácia stojaca za hľadaním modelov na predikciu úpadku firiem je rôzna.
Prvé modely vytvorené na základe dát o verejne obchodovateľných spoločnostiach v USA
boli využívané pri analýzach predchádzajúcich nákupom akcií \cite{altman1968}.
Modely kreditného rizika našli veľké využitie v bankovníctve – banky a iné finančné inštitúcie disponujú internými modelmi,
ktorými vyhodnocujú finančnú kondíciu svojich klientov a ich hodnotenia využívajú napr. pri stanovovaní úrokových sadzieb.
Takisto aj súkromné spoločnosti a živnostníci môžu využiť modely kreditného rizika pri bežnom vykonávaní svojej podnikateľskej praxe.

V prvých dvoch kapitolách sa oboznamujeme s problematikou bankrotu na Slovenku a predstavíme si dva známe modely kreditného rizika – Altmanovo Z-skóre a český index IN05.
Tretia kapitola je teoretickou predprípravou k metodike,
ktorú sme v práci použili na modelovanie predikcie bankrotu v slovenskom prostredí.
Oboznámime sa v nej s logistickou regresiou a tiež s metódou \emph{BACE}, ktorá slúži na výber vhodnej kombinácie parametrov do modelu.
V štvrtej kapitole sa bližšie venujeme metodike praktickej časti práce.

V piatej kapitole dôkladne opíšeme proces vytvorenia modelov predikcie bankrotu
od popisu dát cez technické detaily modelovania až po porovnanie nami vytvorených modelov s Altmanovým Z-skóre a indexom IN05
aj medzi sebou navzájom. Posledná, šiesta kapitola obsahuje diskusiu o interpretácii výstupov bankrotových modelov. 

    \newpage
    \section{Bankrot na Slovensku}
\label{bankruptcy}

Cieľom tejto práce je vytvoriť model na predikciu úpadku firiem na základe jej finančných údajov.
Pre účely modelovania si na začiatok zadefinujme, čo budeme rozumieť pod pojmom úpadok.
Problematikou bankrotu právnických a fyzických osôb na Slovensku sa venuje najmä zákon 7/2005 Z. z. o konkurze a reštrukuturalizácii \cite{zbierkazakonov},
ktorý definuje úpadok nasledovne.

\bigskip
\textit{(1)
Dlžník je v úpadku, ak je platobne neschopný alebo predlžený. Ak dlžník podá návrh na vyhlásenie konkurzu, predpokladá sa, že je v úpadku.}

\textit{(2)
Právnická osoba je platobne neschopná, ak nie je schopná plniť 30 dní po lehote splatnosti aspoň dva peňažné záväzky viac ako jednému veriteľovi. […]}

\textit{(3)
Predlžený je ten, kto je povinný viesť účtovníctvo podľa osobitného predpisu, má viac ako jedného veriteľa a hodnota jeho záväzkov presahuje hodnotu jeho majetku. […]}
\bigskip

Keď je právnická osoba v úpadku, je potrebné vyhlásiť konkurz alebo reštrukturalizáciu.
Konkurzné a reštrukturalizačné konania schvaľuje súd, vďaka čomu poznáme presný dátum ich začiatku, čo sa nám zíde pri modelovaní.

Konkurz znamená speňaženie majetkovej podstaty dlžníka a pomerné uspokojenie jeho veriteľov z tohto majetku.
Konkurz má charakter likvidačného konania a jeho výsledkom je zrušenie a zánik podniku.
V bežnej praxi ide o zdĺhavý proces, ktorý môže trvať aj niekoľko rokov.

Na rozdiel od konkurzu reštrukturalizácia nemá likvidačný charakter a je možné vyhlásiť ju už v čase hroziaceho úpadku.
Cieľom reštrukturalizácie je zachovanie podniku alebo jeho časti a postupné uspokojenie veriteľov spôsobom dohodnutým v reštrukturalizačnom pláne.
Uspokojenie pohľadávok býva v praxi rýchlejšie ako pri konkurze.
V prípade, že reštrukturalizácia bude úspešná, právnická osoba može naďalej pokračovať v podnikateľskej činnosti,
avšak predpokladom jej úspechu je to, že pohľadávky veriteľov sa budú počas ďalšieho fungovania podniku uspokojovať vo vyššej miere ako v prípade konkurzu.

Spoločným znakom konkurzu a reštrukturalizácie je skutočnosť, že dlžník sa nachádza v krízovej ekonomickej situácii.
Pre účely tejto práce budeme \emph{bankrotom} rozumieť začiatok konkurzného alebo reštrukturalizačného konania.

\subsection{Bankrot - vzácna udalosť}

Úpadok alebo bankrot možno považovať v rámci populácie aktívnych firiem za vzácnu udalosť.
Databáza \emph{FinStat}, ktorej dáta budeme využívať v tejto práci, eviduje celkovo 6946 prípadov konkurzného alebo reštrukturalizačného konania za obdobie existencie samostatnej Slovenskej republiky.
Ako uvidíme, nie každé z týchto konaní sa hodí na modelovanie, či už kvôli nedostatku dostupných dát alebo inej príčiny.

Napríklad, z počtu 6946 firiem v úpadku len pre 3730 firiem poznáme presný dátum začiatku konkurzného alebo reštrukturalizačného konania.
Dátum začiatku konania nepoznáme zväčša pre staršie firmy, ktoré zanikli pred rokom 2010, a pre ktoré by sme v mnohých prípadoch aj tak nemali dostatok dát pre úspešné zahrnutie do modelu.

V tabuľke uvádzame počet konkurzov a reštrukturalizácií a počet aktívnych firiem od roku 2011 po rok 2020.

\begin{center}
    \begin{tabular}{ |c|c|c|c|p{3cm}|p{3cm}| }
        \hline
        Rok & Konkurzy & Reštrukturalizácie & Spolu & Celkový počet aktívnych firiem & Proporcia firiem v úpadku \\
        \hline
        2011 & 282 & 21 & 303 & TODO & TODO \\
        \hline
        2012 & 296 & 36 & 332 & & \\
        \hline
        2013 & 318 & 49 & 367 & & \\
        \hline
        2014 & 329 & 56 & 385 & & \\
        \hline
        2015 & 325 & 49 & 374 & & \\
        \hline
        2016 & 222 & 35 & 257 & & \\
        \hline
        2017 & 215 & 22 & 237 & & \\
        \hline
        2018 & 256 & 6 & 262 & & \\
        \hline
        2019 & 254 & 8 & 262 & & \\
        % pozn: v roku 2019 mala jedna firma konanie s kategoriou "ine", ide o firmu s ico 36595098 ... po rucnej kontrole som ju zaradil medzi restrukturalizacie
        \hline
        2020 & 185 & 19 & 204 & & \\
        \hline
    \end{tabular}
\end{center}
\bigskip

Ako vidíme, počet firiem v úpadku sa každoročne nachádza pod hranicou 0,5 \% z aktívnych firiem.
Z uvedeného vyplýva, že pri modelovaní úpadku firiem budeme pracovať so silno nevyváženými dátami (angl. \emph{angl. unbalanced classes}).
Napriek tomu, že bankrot je v populácii firiem vzácny, ide o udalosť, ktorej predpovedanie je zaujímavé pre množstvo strán -
napr. pre manažment firiem, majiteľov firemného kapitálu, poskytovateľov pôžičiek, investorov či poisťovateľov.

TODO: bližší opis bankrotov podľa odvetví, veľkosti firiem, potenciálne niečo o vplyve covid pandémie atď. a pod.,
dá sa tu spraviť veľa deskriptívnej štatistiky

    \newpage
    \section{Známe bankrotné modely}

Riziko úpadku podnikov sa modeluje už desiatky rokov.
Za jeden z prvých formálnych pokusov o štatistickú analýzu úpadku podnikov je považovaná štúdia FitzPatricka z roku 1932,
v ktorej autor pracoval s dátami o 40 firmách (20 v úpadku, 20 zdravých).
Výsledkom FitzPatrickovej štúdie nebol model rizika úpadku, ale analýza jednotlivých finančných ukazovateľov a ich trendov pri prosperujúcich a neprosperujúcich firmách.

Formálnejšie pokusy o modelovanie rizika úpadku firiem začali v 60. rokoch, kedy vznikli napr. modely Beavera, Tamariho, alebo Altmana,
ktorého Z-skóre vytvorené metódou diskriminačnej analýzy patrí dodnes medzi najpoužívanejšie modely kreditného rizika.
Diskriminačná analýza prevažovala vrámci metód používaných na modelovanie úpadku podnikov až do 80. rokov, kedy ju nahradili metódy,
ktoré majú menej matematických požiadaviek na dáta, ako napr. logistická regresia alebo \emph{probit model} \cite{gruszczynski}.

Rozmach moderných technológií, výpočtovej techniky, a ľahší prístup k dátam v 21. storočí umožnili použitie mnohých ďalších metód na modelovanie kreditného rizika.
V literatúre vieme nájsť modely neurónovej siete, hazardné modely, metódu oporného bodu (SVM) atď. (TODO: nájdi viac príkladov aj so zdrojom)

Známe sú aj modely vytvorené pomocou dát o slovenských a českých firmách, napr. bonitné indexy IN vytvorené manželmi Neumaierovými na dátach o českých firmách,
či Binkertov model a model Martina Gulku zo slovenského prostredia. Na tému analýzy kreditného rizika firiem tiež vzniklo niekoľko záverečných prác na FMFI UK \cite{ondrusekova, bohdal}.

\subsection{Altmanovo Z-skóre}

Altmanovo \(Z\)-skóre patrí historicky k najznámejším a najpoužívanejším modelom predikcie bankrotu.
Prvá verzia \(Z\)-skóre bola vytvorená na vzorke \(66\) verejne obchodovateľných firiem, \(33\) prosperujúcich a \(33\) v úpadku\cite{altman1968}.
Išlo o model viacrozmernej diskriminačnej analýzy
\footnote{v anglickej literatúre sa o metóde využitej pri Altmanovom modeli hovorí ako o \emph{multivariate discriminant analysis} (MDA)}
s \(5\) premennými predstavujúcimi finančné údaje o firme.
Vzorku 33 prosperujúcich firiem zvolil Altman tak, že ku každej z \(33\) firiem v úpadku zvolil firmu podobnej veľkosti pôsobiacu v rovnakom odvetví ako daná firma v úpadku.
Pri vzorke firiem v úpadku používal údaje jeden rok pred vyhlásením bankrotu, pri dopárovaných prosperujúcich firmách používal údaje z toho istého roku.

Pôvodná verzia \(Z\)-skóre pracovala s trhovou hodnotou firmy, ktorá je dostupná len pre verejne obchodovateľné spoločnosti. Jej tvar je nasledovný:

\[
    Z = 0.012X_1 + 0.014X_2 + 0.033X_3 + 0.006X_4 + 0.999X_5
\]

kde

\(X_1 = \) pracovný kapitál / aktíva

\(X_2 = \) výsledok hospodárenia minulých rokov (TODO, alebo nerozdelený zisk, angl. retained earnings) / aktíva

\(X_3 = \) EBIT\footnote{zisk pred zdanením a úrokmi (angl. \emph{earnings before interest and taxes})} / aktíva

\(X_4 = \) trhová hodnota / celkové záväzky

\(X_5 = \) tržby / aktíva.

Podľa modelu je spoločnosť v prosperujúcej zóne, ak \(Z > 2.99\), a v úpadku, ak \(Z < 1.81\).
Interval \([1.81, 2.99]\) nazývame „šedou“ zónou – ak má firma hodnotu \(Z\) v danom intervale, model firmu jednoznačne nezaraďuje do žiadnej z dvoch skupín
(prosperujúca firma alebo firma v úpadku).

Ako bolo zdôraznené v \cite{altman1983}, pôvodné \(Z\)-skóre bolo vytvorené na vzorke verejne obchodovateľných spoločností a je aplikovateľné len na tento typ firiem.
Pokusom o \emph{ad hoc} modifikáciu modelu (napr. nahradením trhovej hodnoty vlastným imaním) chýbala vedecká exaktnosť.
V roku 1983 vytvoril Altman novú verziu \(Z\)-skóre, aplikovateľnú aj na súkromné spoločnosti, v tvare:

\[
    Z = 0.717X_1 + 0.847X_2 + 3.107X_3 + 0.420X_4 + 0.998X_5
\]

kde \(X_4 = \) trhová hodnota / celkové záväzky, a ostatné premenné majú rovnaký význam ako v pôvodnom \(Z\)-skóre z roku 1968.

Hranice intervalov pre modifikované \(Z\)-skóre sú:

\( Z > 2.9\) – prosperujúca zóna

\( Z \in [1.2, 2.9]\) – šedá zóna

\( Z < 1.2 \) – zóna úpadku

Keďže väčšina firiem na Slovensku je v súkromnom vlastníctve, v slovenskom prostredí sa využíva hlavne verzia Altmanovho skóre z roku 1983.
Presnosť modelov vytvorených v tejto práci budeme porovnávať práve s touto verziou \(Z\)-skóre.

Aj autor \(Z\)-skóre Edward Altman uznal, že presnosť jeho modelu bola prekonaná inými, konkurenčnými modelmi \cite{altman2017}, napriek tomu však \(Z\)-skóre patrí medzi najpoužívanejšie modely predikcie bankrotu.
Dôvodmi pre túto skutočnosť sú zrejme jednoduchosť a ľahká interpretovateľnosť jeho vzorca a dostatočne vysoká presnosť modelu v globálnom prostredí
(v \cite{altman2017} Altman ukazuje, že presnosť klasifikácie pre väčšinu krajín je vyššia než približne \(0.75\)).


\subsection{Index IN05}

INDEX IN05 je český model slúžiaci na ohodnotenie finančného zdravia spoločnosti.
Model patrí do rodiny bonitných indexov IN vytvorených manželmi Neumaierovými, pričom index IN05 je najnovší z nich a pochádza z roku 2005.
Podobne ako Altmanovo Z-skóre, tento model vznikol na základe diskriminačnej analýzy. Diskriminačná funkcia indexu IN05 má tvar:

\[
    \text{IN05} = 0.13X_1 + 0.04X_2 + 3.97X_3 + 0.21X_4 + 0.09X_5
\]

kde

\(X_1 = \) aktíva / cudzie zdroje

\(X_2 = \) EBIT / nákladové úroky

\(X_3 = \) EBIT / aktíva

\(X_4 = \) výnosy / aktíva

\(X_5 = \) obežné aktíva / (krátkodobé závazky + bežné bankové úvery).

Hranice intervalov pre hodnoty indexu IN05 sú:

\( \text{IN05} > 1.6\) – prosperujúca zóna

\( \text{IN05} \in [0.9, 1.6]\) – šedá zóna

\( \text{IN05} < 0.9 \) – zóna úpadku.

Podľa publikácie \cite{sav} má pri identifikácii bankrotu index IN05 úspešnosť \(77 \%\), ktorá bola nameraná na vzorke \( 1526 \) českých podnikov.



    \newpage  
    \section{Teoretická predpríprava}
\label{teoreticka predpriprava}

\subsection{Logistická regresia}
 
Logistická regresia je štatistická metóda využívaná pri binárnej klasifikácii.
Cieľom logistickej regresie je modelovať pravdepodobnosť nejakej triedy alebo udalosti (vysvetľovanej premennej)
na základe jednej alebo viacerých vysvetľujúcich premenných. TODO

Pred tým, než opíšeme, ako presne logistická regresia funguje, si zadefinujme niekoľko dôležitých pojmov, s ktorými logistická regresia narába.

\begin{defin}
    Logistická funkcia \( \sigma : \mathbb{R} \rightarrow (0, 1) \) je definovaná ako:
    \[
        \sigma(t) = \frac{e^t}{e^t + 1} = \frac{1}{1 + e^{-t}}
    \]
    Inverzná funkcia k logistickej sa nazýva logit funkcia a spĺňa:
    \[
        logit(t) = \sigma^{-1}(t) = \ln\left(\frac{p}{1 - p}\right)
    \]
    pre \( p \in (0, 1) \).
\end{defin}

Na obrázku je zobrazený graf logistickej funkcie na intervale \( (-6, 6) \):

% \begin{center}
%     \begin{tikzpicture}
%     \begin{axis}[
%         xlabel={x},
%         ylabel={y},
%         xmin=-6, xmax=6,
%         ymin=0, ymax=1,
%         xtick={-4,-2,0,2,4},
%         ytick={0,0.2,0.4,0.6,0.8,1},
%         legend pos=north west,
%         ymajorgrids=true,
%         grid style=dashed,
%         height=8cm,
%         width=15cm,
%     ]
    
%     \addplot[
%         color=blue,
%         mark=square,
%         ]
%         coordinates {
%         (2, 1.0)(3, 0.5)(4, 0.25)(5, 0.16667)(6, 0.11111)(7, 0.08333)(8, 0.0625)(9, 0.05)(10, 0.04)(11, 0.03333)(12, 0.02778)(13, 0.02381)(14, 0.02041)(15, 0.01786)(16, 0.015625)
%         };
%         % \legend{Rozptyl}
        
%     \end{axis}
%     \end{tikzpicture}
% \end{center}


\begin{center}
\begin{tikzpicture}[>=stealth]
    \begin{axis}[
        xmin=-6,xmax=6,
        ymin=0,ymax=1,
        axis x line=middle,
        axis y line=middle,
        axis line style=<->,
        xlabel={$x$},
        ylabel={$y$},
        ]
        \addplot[no marks,blue,solid] expression[domain=-6:6,samples=100]{1/(1 + exp(-x))};
    \end{axis}
\end{tikzpicture}
\end{center}

Argumentom prirodzeného logaritmu v logit funkcii je výraz \( \frac{p}{1 - p} \) pre \( p \in (0, 1) \).
Ak takéto \(p\) budeme chápať ako pravdepodobnosť, výraz \( \frac{p}{1 - p} \) predstavuje takzvaný pomer šancí (\emph{angl. odds-ratio}).
Napríklad pri hode kockou je pravdepodobnosť padnutia šestky rovná \( 1/6 \) a pomer šancí je \( \frac{\frac{1}{6}}{1 - \frac{1}{6}} = \frac{\frac{1}{6}}{\frac{5}{6}} = \frac{1}{5} = 1 : 5\),
čo znamená, že 1 možný výsledok hodu kockou zodpovedá šestke a 5 možných výsledkov šestke nezodpovedá.

Nech \( Y \) je binárna vysvetľovaná premenná, ktorej zodpovedá vektor vysvetľujúcich premenných \( x = (x_1, x_2, \ldots, x_k) \).
Pre \( Y \) zjavne platí \( P(Y = 1|x) = 1 - P(Y = 0|x) \).
Logistická regresia predpokladá, že logaritmus pomeru šancí (\emph{angl. log-likelihood ratio}) možno modelovať ako lineárnu funkciu zložiek vektora \( x \).

\[
\ln \left( \frac{P(Y = 1|x)}{P(Y = 0|x)} \right) = \ln \left( \frac{P(Y = 1|x)}{1 - P(Y = 1|x)} \right) = \beta_0 + \beta^T x
\]

Po úpravách sa ľahko dopracujeme k záveru, že pravdepodobnosť \( P(Y = 1|x) \) je rovná výstupu logistickej funkcie, ktorej argument bude hľadaná lineárna kombinácia zložiek vektora \( x \).

\begin{equation} \label{logistic_regression}
P(Y = 1|x) = \frac{1}{1 - e^{-(\beta_0 + \beta^T x})} := h(x)
\end{equation}

Ak logistickú regresiu používame na predikciu (čo budeme robiť neskôr v tejto práci),
vzorec (\ref{logistic_regression}) nám poslúži na výpočet pravdepodobnosti \( P(Y = 1|x) \) pri danom novom \( x \).

\subsubsection{Odhad parametrov v logistickej regresii}

Majme teda zovšebecnený regresný model tvaru

\[
h_\beta(x) = P(Y = 1|x) = \frac{1}{1 - e^{-(\beta_0 + \beta^T x})}
\]

Neznámymi parametrami v tomto modeli sú regresory \( \beta = (\beta_1, \ldots, \beta_k) \).
Na odhad regresných koeficientov sa vo väčšine prípadov používa metóda maximálnej vierohodnosti.
Na rozdiel od obyčajnej lineárnej regresie s normálne rozdelenými chybami, v logistickej regresii nie je možné nájsť exaktné vyjadrenie parametrov \( \beta \),
a na ich odhad sa používa nejaká iteračná metóda.

Keďže \(Y\) nadobúda len hodnoty z \( \{0, 1\} \), pre distribučnú funkciu \(Y\) platí:

\[
P(y | x; \beta ) = h_\beta(x)^y (1 - h_\beta(x))^{1 - y}
\]

Majme namerané vektory dát \( x_1, \ldots, x_n \), \( x_i = x_{i1}, \ldots, x_{ik} \),
a k nim prislúchajúce \( y_1, \ldots, y_n \). Potom pre funkciu vierohodnosti parametra \( \beta \) platí

\[
L(\beta | y; ) TODO
\]

Trénovanie logistickej regresie spočíva v maximalizovaní funkcie vierohodnosti,
čo je ekvivalentné maximalizácii jej logaritmu (\emph{angl. log-likelihood function}), a teda hľadáme

\[
TODO
\]

Na nájdenie maxima log-likelihood funkcie sa používajú iteračné metódy,
napr. funkcia \emph{glm} v základnej verzii jazyka \emph{R} využíva metódu \emph{IRLS} (\emph{iteratively reweighted least squares}).

\subsubsection{Sumárne štatistiky v logistickej regresii}

TODO: Tu napíšem niečo o hodnotách ako \(R^2\) atď., ešte to nemám celkom premyslené.

\subsubsection{Interpretácia parametrov v logistickej regresii}

    \newpage  
    \section{Metodika praktickej časti práce}
\label{metodika}

V praktickej časti diplomovej práce vytvoríme niekoľko modelov logistickej regresie na predikciu bankrotu firiem.
Na modelovanie použijeme finančné dáta o slovenských firmách dostupné z databázy \emph{FinStat}.

Bankrot predstavuje v populácii firiem vzácnu udalosť, preto jedným z problémov bude vybrať vhodnú vzorku na natrénovanie modelov.
Firiem, ktoré za celé obdobie ich existencie neboli v úpadku, je obrovské množstvo, a z tohto množstva použijeme na tréning len malú vzorku – vykonáme tzv. undersampling.
Rôzne metódy undersamplingu v súvislosti s využitím na bankrotné modely sú opísané napr. v \cite{protopapadakis},
zo záverov tejto práce však vyplýva, že komplikovanejšie metódy undersamplingu neprinášajú preukázateľne lepšie výsledky než jednoduchý náhodný výber.

Pre zachovanie jednoduchosti vykonáme undersampling podobným spôsobom, akým sa vykonáva vo väčšine prác venovaných problematike kreditného rizika \cite{zmijewski},
a to dopárovaním jednej prosperujúcej firmy (t.j. firmy, ktorá za celé obdobie svojho pôsobenia nebola v úpadku) ku každej z firiem v úpadku,
pričom dopárovanie vykonáme na základe zhody odvetvia a podobnosti tržieb.

V ďalšej časti sa zameriame na vytvorenie modelu logistickej regresie na predikciu bankrotu firmy.
Počet údajov (parametrov) v databáze \emph{FinStat} dostupných pre každú jednu firmu je vyše 70, nie všetky z týchto parametrov sa ale hodia na predikciu bankrotu.
Na výber najsignifikantnejších ukazovateľov použijeme dve metódy – LASSO (\emph{least absolute shrinkage and selection operator}) a bayesovské priemerovanie modelov.

\subsection{Interpretovateľnosť ako sila logistickej regresie}

Odkedy Edward Altman v roku 1968 uverejnil prvý článok o svojom Z-skóre, bolo na tému predikcie úpadku napísaných desiatky ďalších prác.
Napriek tomu, že mnohé z týchto analýz vyústili do modelov, ktoré mali preukázateľne lepšiu predikčnú schopnosť ako Z-skóre,
Altmanov model diskriminačnej analýzy patrí dodnes medzi najpoužívanejšie a najcitovanejšie modely kreditného rizika [TODO: zdroj?].

Dôvodov pre túto skutočnosť je niekoľko, ale najvýraznejším argumentom pre Altmanovo Z-skóre je jeho relatívna jednoduchosť (pracuje len s 5 údajmi)
a z nej vyplývajúca široká aplikovateľnosť Z-skóre pre mnohé odvetvia či krajiny.
Mnohé z modelov, ktoré preukázali lepšiu predikčnú schopnosť než Z-skóre,
majú omnoho komplikovanejší charakter – pracujú s väčším množstvom parametrov alebo používajú dáta za viac rokov pôsobenia firmy.

Navyše, uvedomme si, že bankrot je silno negatívna udalosť, čo z problematiky jeho predikcie robí výrazne citlivú tému.
V podobných situáciach je často hodnotnejšia predikcia jednoduchším modelom,
ktorého výstup si vie používateľ skontrolovať a zhodnotiť jeho relevanciu pre daný konkrétny vstup, než predikcia komplikovanejším modelom,
ktorý sa síce preukázal ako presnejší na nejakej validačnej vzorke, ale je ťažko interpretovateľný aj pre odborníka z praxe.

Pre účely vytvorenia modelu predikcie úpadku sme zvolili logistickú regresiu práve z toho dôvodu,
že predstavuje kompromis medzi interpretovateľnosťou a predikčnou robustnosťou.
Oproti diskriminačnej analýze, ktorú využil Altman, je model logistickej regresie o niečo komplikovanejší,
na druhej strane ale prináša iné výhody ako napr. zmiernenie matematických požiadaviek na vstupné dáta
či ľahko uchopiteľný obor hôdnot výsledného modelu (interval \([0, 1]\)), a to pri zachovaní jednoduchej interpretovateľnosti jeho parametrov.
Uvedomme si, že logistická regresia nemusí slúžiť len ako nástroj na predikciu a často sa vytvára za účelom štatistickej interpretácie dát.

    \newpage  
    \section{Modelovanie rizika úpadku podnikov}

V tejto kapitole si opíšeme proces vytvorenia modelov logistickej regresie za účelom predikcie rizika úpadku podnikov.

\subsection{Dáta}

V prvom kroku si vytvoríme trénovaciu vzorku pre modely logistickej regresie.
Zdrojom dát je databáza \emph{FinStat}, ktorá spracúva údaje o slovenských spoločnostiach, ktorých je celkovo približne \(250000\).

Niektoré firmy sa nehodia na modelovanie úpadku, či už svojím charakterom alebo nedostatočne či nepresne vyplnenými dátami.
Napríklad budeme pracovať len s firmami, pri ktorých máme dostupné údaje z ich účtovných závierok, pretože ostatné firmy nemajú k dispozícii dostatok finančných údajov.
Podrobný opis podmienok na firmy, ktoré zaradíme do trénovacich dát, uvádzame v nasledovnom zozname:

\begin{enumerate}
    \item Firma musí mať k dispozícii dáta z jej účtovnej závierky. Dôvod: účtovná závierka je zdrojom väčšiny finančných údajov firmy.
    \item Účtovná závierka nesmie byť spracovaná ocr skenom. Dôvod: OCR sken dokumentu niekedy vytiahne nesprávne hodnoty, čo by mohlo negatívne ovplyvniť kvalitu dát.
    \item Právna forma spoločnosti je jedna z nasledovných: Akciová spoločnosť, Družstvo, Komanditná spoločnosť, Spoločnosť s r. o., Verejne obchodovateľná spoločnosť, Jednoduchá spoločnosť na akcie. Vylúčené boli napr. neziskové organizácie, združenia, štátne podniky atď.
    % \item Firma nepôsobí v odvetví s nasledovnými SK NACE kódmi:
    % \begin{itemize}
    %     \item 66290
    %     \item 66200
    %     \item 65120
    %     \item 65110
    %     \item 66000
    %     \item 64190
    %     \item 64100
    %     \item 64200
    %     \item 66110
    %     \item 64300
    %     \item 64910
    % \end{itemize}
    % Ide o spoločnosti z odvetvia poisťovníctva a financií, pri ktorých nie je jednotný spôsob výpočtu finančných údajov a ich zaradenie by mohlo negatívne ovplyvniť kvalitu dát.
    \item Majetok spolu = Spolu vlastné imanie a záväzky (TODO: tomuto nerozumiem)
    \item Obdobie, za ktoré spoločnosť zverejňuje účtovnú závierku, je jeden rok. Túto podmienku spĺňa drvivá väčšina firiem, ale poslúži nám na očistenie od malého množstva spoločností, ktoré účtovné závierky zverejňujú za obdobie inej dĺžky, kvôli čomu ich finančné údaje môžu mať mierne odlišný charakter.
\end{enumerate}

V pôvodných dátach sa nachádzalo celkovo \(3730\) firiem v úpadku s vyplneným dátumom konkurzného alebo reštrukturalizačného konania, z nich \(1145\) spĺňa podmienky 1 až 5.
Všetky z týchto 1145 firiem zaradíme do finálnych dát ako firmy v bankrote.
Uvedomme si, že úpadok budeme modelovať na základe dát za jeden rok (jedno účtovné obdobie), preto do finálneho datasetu zaradíme len dáta rok pred úpadkom firmy.
V ďalšom kroku rozšírime tento dataset o prosperujúce firmy.

\subsection{Prosperujúce firmy}

Prosperujúcou firmou v tejto práci rozumieme firmu, ktorá za celé obdobie svojho pôsobenia nebola v konkurznom ani reštrukturalizačnom konaní, t.j. firmu, ktorá nebola v úpadku.
Prosperujúcich je väčšina firiem na Slovensku a aby sme sa vyhli problému nevyvážených tried (angl. \emph{unbalanced classes}),
na modelovanie rizika úpadku použijeme len časť z nich.
Firmy zvolíme tak, že ku každej z \(1145\) firiem v úpadku z predošlého kroku dopárujeme jednu prosperujúcu firmu podobného charakteru.
Podobná metodika je štandardom pri vytváraní bankrotných modelov – využil ju napr. aj Altmann pri vytváraní prvej verzie svojho Z-skóre – a jej výhodou je aj jej jednoduchosť a pochopiteľnosť, viď podkapitolu \ref{model interpretability}.

Firmy, z ktorých sme vyberali dvojičku pre firmy v úpadku, museli okrem podmienok 1 až 5 z predošlej podkapitoly spĺňať aj nasledovné podmienky:

\begin{enumerate}
    \item Firma má tržby väčšie ako \(5000\) € a celkový majetok vyšší ako \(5000\) €.
    \item Firma má vyplnené všetky finančné údaje, ktoré vstupujú do procesu vytvárania modelu (t.j. všetkých \(70\) finančných údajov, z ktorých len časť bude vystupovať vo finálnom modeli logistickej regresie).
\end{enumerate}

Podmienka 1 bola zvolená za účelom odseparovania firiem, ktoré sú neaktívne alebo príliš drobné.
Hranicu \(5000\) € sme zvolili arbitrárne na základe analýzy veľkosti tržieb, z ktorej vyplynulo, že značná časť firiem má tržby blízke nule a nehodia sa na zaradenie do skupiny prosperujúcich firiem.

Ku každej z \(1145\) firiem v úpadku sme priradili jednu prosperujúcu firmu tým spôsobom,
že spomedzi prosperujúcich firiem pôsobiacich v odvetví s rovnakým SK NACE kódom sme zvolili firmu s čo najpodobnejšími tržbami rok pred úpadkom danej firmy.
Pre \(6\) firiem počas obdobia rok pred ich úpadkom nepôsobila žiadna firma v odvetví s rovnakým SK NACE kódom, preto sme im priradili podobnú prosperujúcu firmu za iný rok.
Pre jednu z týchto firiem neexistovala žiadna firma pôsobiaca v rovnakom odvetví, išlo o firmu s SK NACE kódom 08910 (Ťažba chemických a hnojivových minerálov).
Tejto firme sme priradili firmu z podobného odvetvia s SK NACE kódom 08990 (Iná ťažba a dobývanie i.n.).

Vo finálnom datasete máme dokopy \(2290\) firiem, z toho \(1145\) prosperujúcich a \(1145\) v úpadku.
Pre účely modelovania a následného vyhodnocovania modelov sme tieto dáta rozdelili na trénovaciu, testovaciu a validačnú sadu v pomere \(60:20:20\),
pričom v každej z nich bol pomer prosperujúcich firiem a firiem v úpadku \(50:50\).

V ďalších častiach vytvoríme samotné modely logistickej regresie. Hlavným problémom je vyčlenenie najsignifikantnejších premenných pre predikciu bankrotu.

\subsection{Model selection}

Ku každej z \(2290\) firiem v našich dátach máme k dispozícii \(74\) finančných parametrov.
Cieľom tejto práce je vytvoriť model, ktorý bude obsahovať ideálne len malý počet parametrov, viď podkapitolu \cite{model interpretability}.
V tejto časti sa venujeme výberu najvhodnejšej kombinácie parametrov pre predikciu bankrotu.

\subsubsection{Model LASSO}

Prvou metódou, ktorú sme použili na výber najsignifikantnejších parametrov je \emph{lasso} (\emph{least absolute shrinkage and selection operator}).
Metódu \emph{lasso} sme uprednostili pred inými metódami z podobnej triedy regularizácií (napr. pred \emph{ridge regression}) kvôli skutočnosti,
že metóda \emph{lasso} má väčšiu tendenciu vynulovať parametre, ktoré považuje za nesignifikantné.
(Napr. pri metóde \emph{ridge regression} parametre k nule len konvergujú.)
Pripomíname, že naším cieľom je vytvoriť jednoduchý a interpretovateľný model.

Cieľom metódy \emph{lasso} je nájsť parameter \(\hat{\beta}\) spĺňajúci

\[
    \hat{\beta} = \min_{\beta \in R^p} \left\{ \frac{1}{N} ||y - X \beta||_2^2 + \lambda || \beta ||_2^2 \right\}
\]

Parameter \(\lambda\) určuje stupeň regularizácie – pri vyšších hodnotách \(\lambda\) metóda vynuluje viac parametrov a naopak.
Parameter \(\lambda\) je v metóde \emph{lasso} voľným parametrom, a jeho hodnota bola zvolená použitím krosvalidácie typu \emph{leave one out},
teda pre testovaciu sadu veľkosti \(1832\) (\(916\) firiem v úpadku, \(916\) prosperujúcich) bolo vytvorených \(1832\) regresií,
pričom pri každej bola vylúčená jedna firma, na ktorej sa daná regresia otestovala.

Parametre, ktorým metóda \emph{lasso} pridelila nenulovú hodnotu (a teda ich zaradila do finálneho modelu logistickej regresie) je uvedený v tabuľke \ref{lasso tabulka vsetky parametre}.
Parametrov s nenulovým koeficientom je \(14\).

\begin{table}
    \begin{tabular}{ |c|c| }
        \hline
        EBITDA & Čistý prevádzkový zisk po zdanení (NOPAT) \\
        \hline
        Čistý dlh & Záväzky/EBITDA \\
        \hline
        Celková zadlženosť & Likvidita 1. stupňa\\
        \hline
        Likvidita 3. stupňa & Finančné účty/Aktíva \\
        \hline
        Návratnosť aktív & Obrat aktív \\
        \hline
        Doba splácania záväzkov & INDEX 05 \\
        \hline
        Binkertov model & Spolu majetok (zmena v \%) \\
        \hline
    \end{tabular}
    \caption{Zoznam signifikantných premenných podľa metódy \emph{lasso}}
    \label{lasso tabulka vsetky parametre}
\end{table}

Dodatočnou analýzou sme množinu premenných v modeli okresali ešte viac.
Premenné INDEX 05 a Binkertov model predstavujú hodnotenie finančnej kondície firmy inými modelmi, nepredstavujú teda rýdzo finančný údaj a do modelu sa nehodia.
Premenné Likvidita 1. stupňa a Likvidita 3. stupňa sú vysoko korelované a vybrali sme z nich len Likviditu 3. stupňa,
lebo tej priadil \(t\)-test signifikantnosti parametra výrazne nižšiu \(p\)-hodnotu a teda ju môžeme považovať za signifikantnejšiu premennú.

Vylúčenie ďalších troch parametrov (EBITDA, Čistý prevádzkový zisk po zdanení, a Čistý dlh) vyplynulo z ďalšej analýzy,
pri ktorej sme skúmali výstupy modelu pre širšiu množinu firiem, mimo trénovacej a testovacej sady.
Tieto tri premenné sú hodnoty v eurách a tým, že pri veľkých firmách nadobúdajú tieto hodnoty extrémne veľkosti,
malo to za následok to, že veľkým firmám dávala logistická regresia výstup blízky k \(0\) alebo k \(1\).
Hodnotenia scoringovými modelmi sú často zaujímavé práve pre väčšie firmy, preto sme považovali za vhodnejšie, ak ich hodnotenia budú tvoriť širšie spektrum hodnôt.
Dodatočnú analýzu možného spracovania premenných s extrémnymi hodnotami (napr. zlogaritmovaním) sme nevykonali.

Ďalšie dve premenné, Obrat aktív a Spolu majetok (zmena v \%), sme vylúčili, pretože \(t\)-test signifikantnosti parametra zamietol ich významnosť na hladine \(\alpha = 0.05\).
Konečnú skupinu šiestich premenných, ktoré vstupujú do modelu vytvoreného pomocou metódy \(lasso\), uvádzame spolu s ich koeficientami v tabuľke \ref{lasso tabulka konecne parametre}.

\begin{table}
    \begin{tabular}{ |c|c| }
        \hline
        Premenná & Koeficient \\
        \hline
        Intercept & -0.654131192998752 \\
        \hline
        Záväzky/EBITDA & -0.00184363389130676 \\
        \hline
        Celková zadlženosť & 0.733674242100195 \\
        \hline
        Likvidita 3. stupňa & -0.0603034169898139 \\
        \hline
        Finančné účty/Aktíva & -3.10194708872789 \\
        \hline
        Návratnosť aktív & -1.84060110234586 \\
        \hline
        Doba splácania záväzkov & 0.00000199016251427634 \\
        \hline
    \end{tabular}
    \caption{Konečný zoznam signifikantných premenných podľa metódy \emph{lasso} spolu s ich koeficientami}
    \label{lasso tabulka konecne parametre}
\end{table}

Koeficienty boli spočítané novým natrénovaním modelu na trénovacej sade,
metódu \emph{lasso} sme teda využili len pri skúmaní signifikantnosti jednotlivých premenných.
Vzorec pre výpočet skóre týmto modelom má nasledovný tvar:

\[
    \text{score}_\text{lasso} = \frac{1}{1 + e^{-0.65413 - 0.00184X_1 + 0.73367X_2 - 0.06030X_3 - 3.10194X_4 - 1.84060X_5 + 0.000002X_6}}
\]

kde

\(X_1 = \) Záväzky/EBITDA

\(X_2 = \) Celková zadlženosť

\(X_3 = \) Likvidita 3. stupňa

\(X_4 = \) Finančné účty/Aktíva

\(X_5 = \) Návratnosť aktív

\(X_6 = \) Doba splácania záväzkov.

\subsubsection{Modely BACE}

V ďalšej časti budeme hľadať vhodnú kombináciu premenných modelu predikcie bankrotu metódou \emph{BACE} (\emph{bayesian averaging of classical estimates}).
Metóda \emph{BACE} patrí k heuristickým metódam, ku ktorým možno pristupovať rôznymi spôsobmi, my využijeme dva rôzne prístupy.
Prvý prístup bude replikáciou metodiky využitej v práci \cite{ondrusekova}, v ktorej autorka enumerovala všetky modely logistickej regresie so \(4\), \(5\) a \(6\) parametrami
a z ich výstupov spočítala bayesovské odhady parametra \(\beta\) a jeho aposteriórnu pravdepodobnosť.
Táto práca, rovnako ako naša, sa zaoberala problematikou predikcie bankrotu v slovenskom prostredí.

Druhý prístup k metóde \emph{BACE} vychádza z článku \cite{sala-i-martin}, v ktorom bola metóda \emph{BACE} prvýkrát opísaná.
Pôvodná metóda \emph{BACE} predstavuje algoritmus, ktorý náhodne volí veľké množstvo regresií z množiny regresií so všetkými možnými kombináciami vysvetľujúcich premenných,
až do momentu, kým odhady parametra \(\beta\) neskonvergujú.

\paragraph{Metóda kompletnej enumerácie}

Jediná apriórna informácia, ktorú pri metóde \emph{BACE} vkladáme do výpočtovej procedúry, je očakávaný počet parametrov vo výslednom modeli \( \bar{k} \).
V našom prípade sme zvolili \( \bar{k} = 5 \) na základe niekoľkých skutočností: \(5\) parametrov majú aj zaužívané modely ako Altmanovo Z-skóre či INDEX 05,
a \(5\) parametrov sa ukázalo byť signifikantných (použitím metódy \emph{BACE}) aj v práci \cite{ondrusekova}, ktorú replikujeme.
Zároveň, podľa \cite{sala-i-martin} aj \cite{polaci} malá zmena hodnoty khat nemá veľký vplyv na výstupy metódy \emph{BACE}.
Na základe uvedených skutočností sme usúdili, že voľba \( \bar{k} = 5 \) je pre náš prípad adekvátna.
Dôležitejšie ako konkrétna zvolená hodnota khat je fakt, že \( \bar{k} \) sa pohybuje v nízkych hodnotách a je rádovo menšia oproti celkovému počtu parametrov \( K = 74 \).

V prvej implementácii metódy \emph{BACE} sme kompletne enumerovali všetky modely s \(5\) premennými, zvolenými z celkového počtu \(74\) premenných.
Takých modelov je celkovo \(\binom{74}{5} = 16108764\), z tejto množiny sme však vylúčili tie modely, ktoré obsahovali vysoko korelované (resp. závislé) premenné.
Zoznam skupín závislých premenných uvádzame v prílohe \hyperref[appendix:b]{B}. 
Po vylúčení modelov s korelovanými premennými nám ostalo \(12774840\) modelov, ktoré sme všetky enumerovali.

Po enumerácii koeficientov modelov a ich BIC sme dopočítali aposteriórne pravdepodobnosti modelov a premenných.
Kompletný zoznam premenných s ich apriórnymi a aposteriórnymi pravdepodobnosťami, pri použití metódy BACE prístupom kompletnej enumerácie, uvádzame v prílohe \hyperref[appendix:c]{C}.
V tabuľke \ref{bace1 tabulka pp} uvádzame len tie premenné, ktorých spočítaná aposteriórna pravdepodobnosť zahrnutia bola vyššia ako ich apriórna pravdepodobnosť zahrnutia.
Takých premenných bolo celkovo \(7\).
Uvedomme si, že apriórna pravdepodobnosť zahrnutia premenných nebola v tomto prípade vždy rovná presne \(\frac{5}{74}\),
pretože niektoré modely s \(5\) premennými sme vylúčili, čo malo za následok miernu zmenu hodnoty apriórnej pravdepodobnosti zahrnutia niektorých premenných.

\begin{table}
    \begin{tabular}{ |c|c|c| }
        \hline
        Premenná & \makecell{Apriórna \\ pravdepodobnosť} & \makecell{Aposteriórna \\ pravdepodobnosť} \\
        \hline
        Čistý prevádzkový zisk po zdanení (NOPAT) & 5.9 \% & 98.5 \% \\
        \hline
        Celková zadlženosť & 6.2 \% & 22.8 \% \\
        \hline
        Stupeň samofinancovania & 6.2 \% & 77.2 \% \\
        \hline
        Finančné účty/Aktíva & 7.4 \% & 100 \% \\
        \hline
        Návratnosť aktív & 6.6 \% & 86.2 \%\\
        \hline
        Návratnosť aktív (EBIT) & 6.6 \% & 13.8 \% \\
        \hline
        Doba splácania záväzkov z obchodného styku & 6.6 \% & 100 \% \\
        \hline
    \end{tabular}
    \caption{Konečný zoznam signifikantných premenných podľa metódy \emph{BACE} (kompletná enumerácia modelov s \(5\) premennými) spolu s ich apriórnymi a aposteriórnymi pravdepodobnosťami}
    \label{bace1 tabulka pp}
\end{table}

Dodatočnou analýzou sme pre výsledný model vylúčili \(2\) z týchto \(7\) premenných, \emph{Čistý prevádzkový zisk po zdanení} z rovnakých dôvodov,
z akých sme túto premennú vylúčili pri modeli \emph{lasso}, \emph{Návratnosť aktív (EBIT)} z dôvodu,
že z dvojice vysoko korelovaných premenných spolu s premennou \emph{Návratnosť aktív} mala \emph{Návratnosť aktív (EBIT)} nižšiu aposteriórnu pravdepodobnosť.

Výsledný model bol spočítaný opätovným natrénovaním modelu na trénovacej sade, pri zahrnutí piatich premenných.
Koeficienty pri premenných uvázame v tabuľke ... .

\begin{table}
    \begin{tabular}{ |c|c| }
        \hline
        Premenná & Koeficient \\
        \hline
        Intercept & -0.763142502413478 \\
        \hline
        Celková zadlženosť & 0.706595273475482 \\
        \hline
        Stupeň samofinancovania & -0.0477059620445649 \\
        \hline
        Finančné účty/Aktíva & -3.20026599445248 \\
        \hline
        Návratnosť aktív & -1.86370009476635 \\
        \hline
        Doba splácania záväzkov z obchodného styku & 0.0000653621336106666 \\
        \hline
    \end{tabular}
    \caption{Konečný zoznam signifikantných premenných podľa metódy \emph{BACE} (kompletná enumerácia modelov s \(5\) premennými) spolu s ich koeficientami}
    \label{bace1 tabulka konecne parametre}
\end{table}

Vzorec pre výpočet skóre týmto modelom má nasledovný tvar:

\[
    \text{score}_\text{BACE1} = \frac{1}{1 + e^{-0.76314 + 0.7066X_1 - 0.0477X_2 - 3.20027X_3 - 1.8637X_4 - 1.84060X_5 + 0.000065X_6}}
\]

kde

\(X_1 = \) Celková zadlženosť

\(X_2 = \) Stupeň samofinancovania

\(X_3 = \) Finančné účty/Aktíva

\(X_4 = \) Návratnosť aktív

\(X_5 = \) Doba splácania záväzkov z obchodného styku.

\paragraph{Iteračná metóda}

V druhej implementácii metódy BACE vychádzame z pôvodného článku \cite{sala-i-martin}.
Pri tomto prístupe zvolíme apriórne pravdepodobnosti jednotlivých vysvetľujúcich premenných a budeme náhodne generovať modely logistických regresií,
až kým odhady stredných hodnôt jednotlivých premenných neskonvergujú.
Hodnota apriórnej pravdepodobnosti premenných vychádza, podobne ako v predošlej metóde, z očakávaného počtu premenných vo výslednom modeli.
Opäť sme zvolili očakávaný počet premenných kbar = 5, z čoho vyplýva apriórna pravdepodobnosť zahrnutia pre premenné \( \frac{5}{74} = 0.0676 \).

Okrem apriórnych pravdepodobností jednotlivých premenných má táto procedúra niekoľko hyperparametrov.
Prístup, ktorý sme zvolili, je podobný prístupu z článku \cite{sala-i-martin}, a je nasledovný:
každých \(10000\) iterácií spočítame odhady stredných hodnôt koeficientov pri jednotlivých vysvetľujúcich premenných podľa vzorca TODO,
a keď \(20\)-krát po sebe zmena odhadu stredných hodnôt koeficientov, prenásobených disperziou hodnôt danej premennej v trénovacej sade \(X\),
nepresiahne \(10^{-6}\), algoritmus skončí.
V článku \cite{sala-i-martin} pracujú autori so zastavovacím pravidlom, kedy na prehlásenie konvergencie stačí,
keď sa hodnota koeficientu pri každej z premenných nezmení po \(10000\) iteráciách \(6\)-krát po sebe.
V našom prípade sme zvolili prísnejšie pravidlo, pretože pracujeme s väčšou množinou potenciálnych vysvetľujúcich premenných
(je ich \(74\), v článku \cite{sala-i-martin} ich bolo \(32\)).

Ďalším rozdielom je to, že v článku \cite{sala-i-martin} použili autori apriórnu pravdepodobnosť zahrnutia \( \bar{k}/K \) len pre prvých \(100000\) modeloch,
pri ďalších modeloch používali pri náhodnom výbere odhady aposteriórnych pravdepodobností premenných na základe tejto vzorky \(100000\) modelov
(pričom ich ale ohraničili intervalom \([0.1, 0.85]\)).
Dôvodom bola skutočnosť, že pri takomto prístupe konverguje algoritmus rýchlejšie.
V našej práci sme pre zachovanie jednoduchosti túto metódu nepoužili.
Podobný prístup si vyžaduje voľbu viacerých hyperparametrov, ktoré majú viacmenej arbitrárny charakter, a tým by si sa celý proces skomplikoval.
Navyše, algoritmus by mal skonvergovať pri akejkoľvek voľbe apriórnych pravdepodobností.

Na implementáciu iteračnej metódy \emph{BACE} sme použili softvér \emph{R}, výpočtovým aspektom metódy sa venujeme bližšie v prílohe \hyperref[appendix:d]{D}.

Algoritmus skonvergoval po [TODO] iteráciách.
V tabuľke [TODO] uvádzame aposteriórne pravdepodobnosti zahrnutia a odhady stredných hodnôt koeficientov pre tie premenné,
pre ktoré aposteriórna pravdepodobnosť bola vyššia ako ich apriórna pravdepodobnosť zahrnutia. Kompletný zoznam premenných uvádzame v prílohe E (TODO).

TODO: tabulka

Výsledný model logistickej regresie vytvorený touto metódou je daný vzorcom:

TODO: vzorec

\subsection{Porovnanie modelov}

\begin{tikzpicture}
    \begin{axis}[
      title={ROC krivky},
      xlabel={\(1 -\) špecificita (\emph{false positive rate})},
      ylabel={senzitivita (\emph{true positive rate})},
      legend pos=outer north east,
    ]
    \addplot [blue] table [x=V1, y=V2, col sep=comma] {data/roc_altman.csv};
    \addlegendentry{Altmanovo Z-skóre}

    \addplot [red] table [x=V1, y=V2, col sep=comma] {data/roc_index.csv};
    \addlegendentry{Index IN05}
    \end{axis}
\end{tikzpicture}

    \newpage  
    \section{Interpretácia výstupov modelov}

TODO

	\newpage
  \section*{Záver}
    \addcontentsline{toc}{section}{Záver}
    \markboth{ZÁVER}{ZÁVER} 
    Cieľom práce bolo vytvoriť modely logistickej regresie určené na predikciu bankrotu v slovenskom prostredí.
Na voľbu vhodných kombinácií parametrov sme použili dve metódy – \emph{lasso} a \emph{BACE} (\emph{Bayesian averaging of classical estimates}).
Výsledky naznačujú, že naše modely majú v slovenskom prostredí lepšiu predikčnú schopnosť ako zaužívané modely Altmanovo Z-skóre a index IN05.

Pri vytváraní modelov sme kládli vysoký dôraz na jednoduchosť a ľahkú interpretovateľnosť výsledných vzorcov.
Ukazuje sa, že metóda \emph{BACE}, ktorá medzi štatistikmi pravdepodobne nie je taká známa ako iné metódy voľby modelu,
môže byť vhodnou metódou práve pri problémoch, kde výskumník dbá na interpretovateľnosť a pochopiteľnosť výsledkov.
Výstupom metódy sú okrem iného aj aposteriórne pravdepodobnosti zahrnutia jednotlivých vysvetľujúcich premenných do „správneho“ modelu,
a práve tieto hodnoty môžu mať pre skúmanú oblasť vysokú výpovednú hodnotu.

V ďalšom výskume by sme sa mohli zamerať na rigoróznejšiu analýzu metódy \emph{BACE}, napr. porovnanie dvoch prístupov k jej implementácii
(kompletná enumerácia vopred danej časti modelov, resp. iteračná metóda náhodne generujúca nové a nové modely).
Výsledky našej práce naznačujú, že ich výstupy sú veľmi podobné.
Zaujímavou témou ďalšieho výskumu môže byť porovnanie dvoch implementácií metódy \emph{BACE} v rôznych simulovaných prostrediach.

Modely vytvorené v našej práci nadobúdajú jej zverejnením verejný charakter a môžu byť voľne využité inštitúciami i
jednotlivcami na ohodnotenie kreditného rizika slovenských firiem.
Aj z tohto dôvodu sme počas celého výskumu dávali veľký dôraz na to, aby procesy vytvorenia modelov boli čo najtransparentnejšie
a aby výsledné vzorce mohli byť ľahko interpretovateľné.
Máme za to, že v niektorých situáciách je vhodnejšie nesústrediť sa len na predikčnú schopnosť matematických modelov,
ktoré vytvárame, ale dbať aj na to, aby každý jeden výstup nášho modelu bol obhájiteľný.
To často neplatí pri využití niektorých modernejších štatistických metód, ktorých výstupom býva model tzv. „čiernej skrinky“,
ktorý síce môže mať vyššiu predikčnú schopnosť, ale jeho interné fungovanie nemusí byť navonok jasné.
Uvedomme si, že sme to my, matematici, ktorí v konečnom dôsledku nesieme zodpovednosť za výstupy našich modelov.
Nech aj to je mementom pre výskumníkov, ktorí si vezmú za cieľ vytvorenie verejných matematických modelov.
  
  \newpage
  \renewcommand{\refname}{Zoznam použitej literatúry}
  \begin{thebibliography}{99}

    \bibitem{altman1968} ALTMAN E.: Financial Ratios, Discriminant Analysis and the Prediction of Corporate Bankruptcy. 1968. \emph{Journal of Finance}, 23, 589-609. 

    \bibitem{altman1983} ALTMAN E.: Corporate financial distress: a complete guide to predicting, avoiding, and dealing with bankruptcy. 1983.

    \bibitem{altman2017} ALTMAN E., IWANICZ-DROZDOWSKA M., LAITINEN E., SUVAS A.:
    Financial Distress Prediction in an International Context: A Review and Empirical Analysis of Altman's Z-Score Model. 2017.
    \emph{J Int Financ Manage Account}, 28: 131-171. https://doi.org/10.1111/jifm.12053

    \bibitem{sala-i-martin} SALA-I-MARTIN X., DOPPELHOFER G., MILLER R. I.: Determinants of Long-Term Growth: A Bayesian Averaging of Classical Estimates (BACE) Approach. 2004.
    \emph{The American Economic Review} 94(4), 813–35, http://www.jstor.org/stable/3592794

    \bibitem{gruszczynski} GRUSZCZYŃSKI, M.: On Unbalanced Sampling in Bankruptcy Prediction. \emph{International Journal of Financial Studies.} 2019; 7(2):28. https://doi.org/10.3390/ijfs7020028

    \bibitem{ondrusekova} ONDRUŠEKOVÁ, M.: Modely rizika úpadku podnikov. 2018. Diplomová práca. FMFI UK, Bratislava
    
    \bibitem{bohdal} BOHDAL, M.: Analýza bankrotov pomocou metód strojového učenia. 2020. Bakalárska práca. FMFI UK, Bratislava

    \bibitem{protopapadakis} PROTOPAPADAKIS E., NIKLIS D., DOUMPOS M., DOULAMIS A., ZOPOUNIDIS C.: Sample selection algorithms for credit risk modelling through data mining techniques. 2019.
    \emph{International Journal of Data Mining, Modelling and Management.} 11. 103. 10.1504/IJDMMM.2019.098969.

    \bibitem{zmijewski} ZMIJEWSKI M.: Methodological Issues Related to the Estimation of Financial Distress Prediction Models. 1984. \emph{Journal of Accounting Research}, 22, 59–82. https://doi.org/10.2307/2490859

    \bibitem{carlin} CARLIN B. P., LOUIS T. A.: Bayesian Methods for Data Analysis (3rd ed.). 2008. \emph{Chapman and Hall/CRC}. https://doi.org/10.1201/b14884

    \bibitem{tiao} BOX G. E., TIAO G. C.: Bayesian inference in statistical analysis. 1973. \emph{International Statistical Review}, 43, 242.

    \bibitem{polaci} BIAŁOWOLSKI P., KUSZEWSKI T., WITKOWSKI B.: Macroeconomic Forecasts in Models with Bayesian Averaging of Classical Estimates.
    2012. \emph{Contemporary Economics}. 6. 60. 10.5709/ce.1897-9254.34.

    \bibitem{neumaierova} NEUMAIEROVÁ I., NEUMAIER. I.: Proč se ujal index IN a nikoli pyramidový systém ukazatelů INFA. 2009.

    \bibitem{sav} HARUMOVÁ A., JANISOVÁ M.: Hodnotenie slovenských podnikov pomocou skóringovej funkcie. 2014. \emph{Ekonomický časopis}, 62, č. 5, s. 522 - 539.

    \bibitem{boda} BOĎA M., ÚRADNÍČEK V.: The portability of altman’s Z-score model to predicting corporate financial distress of Slovak companies. 2016.
    \emph{Technological and Economic Development of Economy.} 22. 532-553. 10.3846/20294913.2016.1197165.

    \bibitem{begley} BEGLEY J., MING J., WATTS S.: Bankruptcy Classification Errors in the 1980s: An Empirical Analysis of Altman's and Ohlson's Models.
    1996. \emph{Review of Accounting Studies.} 1. 267-284. 10.1007/BF00570833.

    \bibitem{wu} WU Y., GAUNT C., GRAY S.: A Comparison of Alternative Bankruptcy Prediction Models. 
    2010. \emph{Journal of Contemporary Accounting and Economics.} 6. 10.1016/j.jcae.2010.04.002.

    \bibitem{mcculloch} GEORGE E., MCCULLOCH R.: Approaches for Bayesian Variable Selection. 1997. \emph{Statistica Sinica.} 7. 339-373.

    \bibitem{jamespress} JAMES PRESS S.: Subjective and Objective Bayesian Statistics: principles, models, and applications. 2009. \emph{John Wiley \& Sons, Inc.} 10.1002/9780470317105

    \bibitem{glmnet} HASTIE T., JUNYANG Q., KENNETH T.: An Introduction to glmnet. Dostupné na: \url{https://glmnet.stanford.edu/articles/glmnet.html}

    \bibitem{skogsvik} SKOGSVIK K., SKOGSVIK S.: On the choice based sample bias in probabilistic bankruptcy Prediction. 2013.
    \emph{Investment Management and Financial Innovations.} 10. 29-37.

    \bibitem{king} KING G., ZENG L.: Logistic Regression in Rare Events Data. 2001. \emph{Political Analysis.} 9(2). 137-163. doi:10.1093/oxfordjournals.pan.a004868
    
    \bibitem{bodle}  BODLE K. A., CYBINSKI P. J.,MONEM R.: Effect of IFRS adoption on financial reporting quality: Evidence from bankruptcy prediction.
    2016. \emph{Accounting Research Journal, Vol. 29 No. 3.} s. 292-312. https://doi.org/10.1108/ARJ-03-2014-0029

    \bibitem{manski} MANSKI CH. F., LERMAN S. R.: The Estimation of Choice Probabilities from Choice Based Samples. 1977. \emph{Econometrica 45, č. 8} https://doi.org/10.2307/1914121.

    \bibitem{bishop} BISHOP Y., FIENBERG S., HOLLAND P., LIGHT R., MOSTELLER F.: Discrete Multivariate Analysis: Theory and Practice. 1977. \emph{Applied Psychological Measurement.} 1. 10.1177/014662167700100218.

    \bibitem{anderson} ANDERSON J. A.: Separate Sample Logistic Discrimination. 1972. \emph{Biometrika 59, 4. 1}. 19–35. https://doi.org/10.2307/2334611.

    \bibitem{maddala} MADDALA G. S.: Limited-dependent and qualitative variables in econometrics. 1983. \emph{Cambridge [Cambridgeshire]: Cambridge University Press.}

\end{thebibliography}

  
	\newpage
  \appendix
  \section*{Príloha A} 
    \label{appendix:a}
    \addcontentsline{toc}{section}{Príloha A}
    \markboth{Príloha A}{Príloha A}
    \begin{longtable}{ |c|p{7cm}|c|p{2.4cm}| }
    % \begin{tabular}{ |c|c|c|c| }
        \hline
        Index & Premenná & Typ hodnoty & Preferovaná hodnota \endhead
        \hline
        1 & Tržby spolu & € & vyššia \\
        \hline
        2 & Tržby vrátane tržieb z predaja Dlhodobého majetku a Cenných papierov a podielov & € & vyššia \\
        \hline
        3 & Zisk pred zdanením a úrokmi (EBIT) & € & vyššia \\
        \hline
        4 & EBITDA & € & vyššia \\
        \hline
        5 & Tržby očistené o Zásoby a Aktiváciu & € & vyššia \\
        \hline
        6 & Účtovný cash flow & € & nižšia \\
        \hline
        7 & Náklady na predaný tovar a služby (COGS) & € & nižšia \\
        \hline
        8 & Hrubá tvorba zdrojov z prevádzkovej činnosti & € & vyššia \\
        \hline
        9 & Čistý prevádzkový zisk po zdanení (NOPAT) & € & vyššia \\
        \hline
        10 & Čistý cash flow & € & vyššia \\
        \hline
        11 & Čistý dlh & € & nižšia \\
        \hline
        12 & Hrubá marža & percento & vyššia \\
        \hline
        13 & EBITDA marža & percento & vyššia \\
        \hline
        14 & Prevádzková marža & percento & vyššia \\
        \hline
        15 & Zisková marža & percento & vyššia \\
        \hline
        16 & Marža účtovného cash flow & percento & vyššia \\
        \hline
        17 & Záväzky/EBITDA & des2 & nižšia \\
        \hline
        18 & Celková zadlženosť & percento & ani vysoká, ani nízka \\
        \hline
        19 & Stupeň samofinancovania & percento & ani vysoká, ani nízka \\
        \hline
        20 & Krátkodobá zadlženosť & percento & ani vysoká, ani nízka \\
        \hline
        21 & Finančná páka & des2 & ani vysoká, ani nízka \\
        \hline
        22 & Platobná neschopnosť celková & číslo & nižšia \\
        \hline
        23 & Likvidita 1. stupňa & des2 & vyššia \\
        \hline
        24 & Likvidita 2. stupňa & des2 & vyššia \\
        \hline
        25 & Likvidita 3. stupňa & des2 & vyššia \\
        \hline
        26 & Finančné účty/Aktíva & percento & ani vysoká, ani nízka \\
        \hline
        27 & Návratnosť aktív & percento & vyššia \\
        \hline
        28 & Návratnosť aktív (EBIT) & percento & vyššia \\
        \hline
        29 & Návratnosť dlhodobého kapitálu (EBIT) & percento & vyššia \\
        \hline
        30 & Doba obratu aktív & dni & nižšia \\
        \hline
        31 & Obrat aktív & číslo & vyššia \\
        \hline
        32 & Obrat obežného majetku & číslo & vyššia \\
        \hline
        33 & Doba splácania záväzkov & dni & ani vysoká, ani nízka \\
        \hline
        34 & Doba splácania záväzkov vo vzťahu k tržbám & dni & ani vysoká, ani nízka \\
        \hline
        35 & Doba splácania záväzkov z obchodného styku & dni & ani vysoká, ani nízka \\
        \hline
        36 & Prirážka & percento & vyššia \\
        \hline
        37 & Finančná efektívnosť tržieb & číslo & vyššia \\
        \hline
        38 & Doba splácania dlhov z čistého cash flow & rok & nižšia \\
        \hline
        39 & Stupeň oddlženia & percento & vyššia \\
        \hline
        40 & Krátkodobá toková oddlženosť & percento & vyššia \\
        \hline
        41 & Miera zadlženosti vlastného imania & percento & vyššia \\
        \hline
        42 & Altmanovo Z-skóre & des2 & vyššia \\
        \hline
        43 & INDEX 05 & des2 & vyššia \\
        \hline
        44 & Binkertov model & des2 & vyššia \\
        \hline
        45 & In. Bonity & des2 & vyššia \\
        \hline
        46 & Tafflerov model & des2 & vyššia \\
        \hline
        47 & Spolu majetok (zmena v \%) & percento &  \\
        \hline
        48 & Obežný majetok (zmena v \%) & percento &  \\
        \hline
        49 & Krátkodobé pohľadávky súčet (zmena v \%) & percento &  \\
        \hline
        50 & Pohľadávky z obchodného styku (zmena v \%) & percento &  \\
        \hline
        51 & Finančné účty súčet (zmena v \%) & percento &  \\
        \hline
        52 & Peniaze (zmena v \%) & percento &  \\
        \hline
        53 & Účty v bankách (zmena v \%) & percento &  \\
        \hline
        54 & Spolu vlastné imanie a záväzky (zmena v \%) & percento &  \\
        \hline
        55 & Vlastné imanie (zmena v \%) & percento &  \\
        \hline
        56 & Základné imanie súčet (zmena v \%) & percento &  \\
        \hline
        57 & Výsledok hospodárenia minulých rokov (zmena v \%) & percento &  \\
        \hline
        58 & Výsledok hospodárenia za účtovné obdobie po zdanení (zmena v \%) & percento &  \\
        \hline
        59 & Záväzky (zmena v \%) & percento &  \\
        \hline
        60 & Krátkodobé záväzky súčet (zmena v \%) & percento &  \\
        \hline
        61 & Závazky z obchodného styku (zmena v \%) & percento &  \\
        \hline
        62 & Daňové závazky a dotácie (zmena v \%) & percento &  \\
        \hline
        63 & Výrobná spotreba (zmena v \%) & percento &  \\
        \hline
        64 & Spotreba materiálu, energie a ostatných neskladovateľný dodávok (zmena v \%) & percento &  \\
        \hline
        65 & Služby (zmena v \%) & percento &  \\
        \hline
        66 & Pridaná hodnota (zmena v \%) & percento &  \\
        \hline
        67 & Výsledok hospodárenia z hospodárskej činnosti (zmena v \%) & percento &  \\
        \hline
        68 & Ostatné náklady na finančnú činnosť (zmena v \%) & percento &  \\
        \hline
        69 & Výsledok hospodárenia z finančnej činnosti (zmena v \%) & percento &  \\
        \hline
        70 & Výsledok hospodárenia z bežnej činnosti pred zdanením (zmena v \%) & percento &  \\
        \hline
        71 & Výsledok hospodárenia z bežnej činnosti po zdanení (zmena v \%) & percento &  \\
        \hline
        72 & Výsledok hospodárenia za účtovné obdobie pred zdanením (zmena v \%) & percento &  \\
        \hline
        73 & Výsledok hospodárenia za účtovné obdobie po zdanení (zmena v \%) & percento &  \\
        \hline
        74 & Tržby spolu (zmena v \%) & percento &  \\
        \hline
    % \end{tabular}
    \caption{Zoznam všetkých premenných}
    \label{zoznam vsetkych premennych}
\end{longtable}

  \newpage
  \section*{Príloha B} 
    \label{appendix:b}
    \addcontentsline{toc}{section}{Príloha B}
    \markboth{Príloha B}{Príloha B}
    \begin{longtable}{ |c| }
    % \begin{tabular}{ |c|c|c|c| }
        % \hline
        % Index & Premenná & Typ hodnoty & Preferovaná hodnota \endhead
        \hline
        Tržby spolu \\
        Tržby vrátane tržieb z predaja Dlhodobého majetku a Cenných papierov a podielov \\
        Tržby očistené o Zásoby a Aktiváciu \\
        \hline
        Zisk pred zdanením a úrokmi (EBIT) \\
        EBITDA \\
        \hline
        Náklady na predaný tovar a služby (COGS) \\
        \hline
        Účtovný cash flow \\
        Hrubá tvorba zdrojov z prevádzkovej činnosti \\
        Čistý prevádzkový zisk po zdanení (NOPAT) \\
        Čistý cash flow \\
        Čistý dlh \\
        \hline
        Hrubá marža \\
        EBITDA marža \\
        Prevádzková marža \\
        Zisková marža \\
        Marža účtovného cash flow \\
        \hline
        Záväzky/EBITDA \\
        \hline
        Celková zadlženosť \\
        Stupeň samofinancovania \\
        Krátkodobá zadlženosť \\
        Finančná páka \\
        \hline
        Platobná neschopnosť celková \\
        \hline
        Likvidita 1. stupňa \\
        Likvidita 2. stupňa \\
        Likvidita 3. stupňa \\
        \hline
        Finančné účty/Aktíva \\
        \hline
        Návratnosť aktív \\
        Návratnosť aktív (EBIT) \\
        Návratnosť dlhodobého kapitálu (EBIT) \\
        \hline
        Doba obratu aktív \\
        Obrat aktív \\
        Obrat obežného majetku \\
        \hline
        Doba splácania záväzkov \\
        Doba splácania záväzkov vo vzťahu k tržbám \\
        Doba splácania záväzkov z obchodného styku \\
        \hline
        Prirážka \\
        \hline
        Finančná efektívnosť tržieb \\
        \hline
        Doba splácania dlhov z čistého cash flow \\
        Stupeň oddlženia \\
        Krátkodobá toková oddlženosť \\
        Miera zadlženosti vlastného imania \\
        \hline
        Altmanovo Z-skóre \\
        \hline
        INDEX 05 \\
        \hline
        Binkertov model \\
        \hline
        In. Bonity \\
        \hline
        Tafflerov model \\
        \hline
        Spolu majetok (zmena v \%) \\
        \hline
        Obežný majetok (zmena v \%) \\
        \hline
        Krátkodobé pohľadávky súčet (zmena v \%) \\
        \hline
        Pohľadávky z obchodného styku (zmena v \%) \\
        \hline
        Finančné účty súčet (zmena v \%) \\
        Peniaze (zmena v \%) \\
        Účty v bankách (zmena v \%) \\
        \hline
        Spolu vlastné imanie a záväzky (zmena v \%) \\
        \hline
        Vlastné imanie (zmena v \%) \\
        Základné imanie súčet (zmena v \%) \\
        Výsledok hospodárenia minulých rokov (zmena v \%) \\
        Výsledok hospodárenia za účtovné obdobie po zdanení (zmena v \%) \\
        \hline
        Záväzky (zmena v \%) \\
        \hline
        Krátkodobé záväzky súčet (zmena v \%) \\
        Závazky z obchodného styku (zmena v \%) \\
        Daňové závazky a dotácie (zmena v \%) \\
        \hline
        Výrobná spotreba (zmena v \%) \\
        Spotreba materiálu, energie a ostatných neskladovateľný dodávok (zmena v \%) \\
        \hline
        Služby (zmena v \%) \\
        \hline
        Pridaná hodnota (zmena v \%) \\
        \hline
        Výsledok hospodárenia z hospodárskej činnosti (zmena v \%) \\
        \hline
        Ostatné náklady na finančnú činnosť (zmena v \%) \\
        \hline
        Výsledok hospodárenia z finančnej činnosti (zmena v \%) \\
        \hline
        Výsledok hospodárenia z bežnej činnosti pred zdanením (zmena v \%) \\
        \hline
        Výsledok hospodárenia z bežnej činnosti po zdanení (zmena v \%) \\
        \hline
        Výsledok hospodárenia za účtovné obdobie pred zdanením (zmena v \%) \\
        \hline
        Výsledok hospodárenia za účtovné obdobie po zdanení (zmena v \%) \\
        \hline
        Tržby spolu (zmena v \%) \\
        \hline        
    % \end{tabular}
    \caption{Skupiny korelovaných premenných}
    \label{skupiny korelovanych premennych}
\end{longtable}

  \newpage
  \section*{Príloha C} 
    \label{appendix:c}
    \addcontentsline{toc}{section}{Príloha C}
    \markboth{Príloha C}{Príloha C}
    \begin{longtable}{ |c|p{7cm}|c|c| }
    % \begin{tabular}{ |c|c|c|c| }
        \hline
        Index & Premenná & \makecell{Apriórna \\ pravdepodobnosť} & \makecell{Aposteriórna \\ pravdepodobnosť} \endhead
        \hline
        1 & Tržby spolu & 6.6 \% & 0 \% \\
        \hline
        2 & Tržby vrátane tržieb z predaja Dlhodobého majetku a Cenných papierov a podielov & 6.6 \% & 0 \% \\
        \hline
        3 & Zisk pred zdanením a úrokmi (EBIT) & 7.0 \% & 0.6 \% \\
        \hline
        4 & EBITDA & 7.0 \% & 0 \% \\
        \hline
        5 & Tržby očistené o Zásoby a Aktiváciu & 6.6 \% & 0 \% \\
        \hline
        6 & Účtovný cash flow & 5.9 \% & 0 \% \\
        \hline
        7 & Náklady na predaný tovar a služby (COGS) & 7.4 \% & 0 \% \\
        \hline
        8 & Hrubá tvorba zdrojov z prevádzkovej činnosti & 5.9 \% & 0 \% \\
        \hline
        9 & Čistý prevádzkový zisk po zdanení (NOPAT) & 5.9 \% & 98.5 \% \\
        \hline
        10 & Čistý cash flow & 5.9 \% & 0 \% \\
        \hline
        11 & Čistý dlh & 5.9 \% & 0.9 \% \\
        \hline
        12 & Hrubá marža & 5.9 \% & 0 \% \\
        \hline
        13 & EBITDA marža & 5.9 \% & 0 \% \\
        \hline
        14 & Prevádzková marža & 5.9 \% & 0 \% \\
        \hline
        15 & Zisková marža & 5.9 \% & 0 \% \\
        \hline
        16 & Marža účtovného cash flow & 5.9 \% & 0 \% \\
        \hline
        17 & Záväzky/EBITDA & 7.4 \% & 0 \% \\
        \hline
        18 & Celková zadlženosť & 6.2 \% & 22.8 \% \\
        \hline
        19 & Stupeň samofinancovania & 6.2 \% & 77.2 \% \\
        \hline
        20 & Krátkodobá zadlženosť & 6.2 \% & 0 \% \\
        \hline
        21 & Finančná páka & 6.2 \% & 0 \% \\
        \hline
        22 & Platobná neschopnosť celková & 7.4 \% & 0 \% \\
        \hline
        23 & Likvidita 1. stupňa & 6.6 \% & 0 \% \\
        \hline
        24 & Likvidita 2. stupňa & 6.6 \% & 0 \% \\
        \hline
        25 & Likvidita 3. stupňa & 6.6 \% & 0 \% \\
        \hline
        26 & Finančné účty/Aktíva & 7.4 \% & 100 \% \\
        \hline
        27 & Návratnosť aktív & 6.6 \% & 86.2 \% \\
        \hline
        28 & Návratnosť aktív (EBIT) & 6.6 \% & 13.8 \% \\
        \hline
        29 & Návratnosť dlhodobého kapitálu (EBIT) & 6.6 \% & 0 \% \\
        \hline
        30 & Doba obratu aktív & 6.6 \% & 0 \% \\
        \hline
        31 & Obrat aktív & 6.6 \% & 0 \% \\
        \hline
        32 & Obrat obežného majetku & 6.6 \% & 0 \% \\
        \hline
        33 & Doba splácania záväzkov & 6.6 \% & 0 \% \\
        \hline
        34 & Doba splácania záväzkov vo vzťahu k tržbám & 6.6 \% & 0 \% \\
        \hline
        35 & Doba splácania záväzkov z obchodného styku & 6.6 \% & 100 \% \\
        \hline
        36 & Prirážka & 7.4 \% & 0 \% \\
        \hline
        37 & Finančná efektívnosť tržieb & 7.4 \% & 0 \% \\
        \hline
        38 & Doba splácania dlhov z čistého cash flow & 6.2 \% & 0 \% \\
        \hline
        39 & Stupeň oddlženia & 6.2 \% & 0 \% \\
        \hline
        40 & Krátkodobá toková oddlženosť & 6.2 \% & 0 \% \\
        \hline
        41 & Miera zadlženosti vlastného imania & 6.2 \% & 0 \% \\
        \hline
        42 & Altmanovo Z-skóre & 7.4 \% & 0 \% \\
        \hline
        43 & INDEX 05 & 7.4 \% & 0 \% \\
        \hline
        44 & Binkertov model & 7.4 \% & 0 \% \\
        \hline
        45 & In. Bonity & 7.4 \% & 0 \% \\
        \hline
        46 & Tafflerov model & 7.4 \% & 0 \% \\
        \hline
        47 & Spolu majetok (zmena v \%) & 7.4 \% & 0 \% \\
        \hline
        48 & Obežný majetok (zmena v \%) & 7.4 \% & 0 \% \\
        \hline
        49 & Krátkodobé pohľadávky súčet (zmena v \%) & 7.4 \% & 0 \% \\
        \hline
        50 & Pohľadávky z obchodného styku (zmena v \%) & 7.4 \% & 0 \% \\
        \hline
        51 & Finančné účty súčet (zmena v \%) & 6.6 \% & 0 \% \\
        \hline
        52 & Peniaze (zmena v \%) & 6.6 \% & 0 \% \\
        \hline
        53 & Účty v bankách (zmena v \%) & 6.6 \% & 0 \% \\
        \hline
        54 & Spolu vlastné imanie a záväzky (zmena v \%) & 7.4 \% & 0 \% \\
        \hline
        55 & Vlastné imanie (zmena v \%) & 6.2 \% & 0 \% \\
        \hline
        56 & Základné imanie súčet (zmena v \%) & 6.2 \% & 0 \% \\
        \hline
        57 & Výsledok hospodárenia minulých rokov (zmena v \%) & 6.2 \% & 0 \% \\
        \hline
        58 & Výsledok hospodárenia za účtovné obdobie po zdanení (zmena v \%) & 6.2 \% & 0 \% \\
        \hline
        59 & Záväzky (zmena v \%) & 7.4 \% & 0 \% \\
        \hline
        60 & Krátkodobé záväzky súčet (zmena v \%) & 6.6 \% & 0 \% \\
        \hline
        61 & Závazky z obchodného styku (zmena v \%) & 6.6 \% & 0 \% \\
        \hline
        62 & Daňové závazky a dotácie (zmena v \%) & 6.6 \% & 0 \% \\
        \hline
        63 & Výrobná spotreba (zmena v \%) & 7.0 \% & 0 \% \\
        \hline
        64 & Spotreba materiálu, energie a ostatných neskladovateľný dodávok (zmena v \%) & 7.0 \% & 0 \% \\
        \hline
        65 & Služby (zmena v \%) & 7.4 \% & 0 \% \\
        \hline
        66 & Pridaná hodnota (zmena v \%) & 7.4 \% & 0 \% \\
        \hline
        67 & Výsledok hospodárenia z hospodárskej činnosti (zmena v \%) & 7.4 \% & 0 \% \\
        \hline
        68 & Ostatné náklady na finančnú činnosť (zmena v \%) & 7.4 \% & 0 \% \\
        \hline
        69 & Výsledok hospodárenia z finančnej činnosti (zmena v \%) & 7.4 \% & 0 \% \\
        \hline
        70 & Výsledok hospodárenia z bežnej činnosti pred zdanením (zmena v \%) & 7.4 \% & 0 \% \\
        \hline
        71 & Výsledok hospodárenia z bežnej činnosti po zdanení (zmena v \%) & 7.4 \% & 0 \% \\
        \hline
        72 & Výsledok hospodárenia za účtovné obdobie pred zdanením (zmena v \%) & 7.4 \% & 0 \% \\
        \hline
        73 & Výsledok hospodárenia za účtovné obdobie po zdanení (zmena v \%) & 7.4 \% & 0 \% \\
        \hline
        74 & Tržby spolu (zmena v \%) & 7.4 \% & 0 \% \\
        \hline
    % \end{tabular}
    \caption{Konečný zoznam všetkých premenných spolu s ich apriórnymi a aposteriórnymi pravdepodobnosťami (metóda \emph{BACE}: kompletná enumerácia modelov s \(5\) premennými)}
    \label{bace1_pp}
\end{longtable}

  \newpage
  \section*{Príloha D} 
    \label{appendix:d}
    \addcontentsline{toc}{section}{Príloha D}
    \markboth{Príloha D}{Príloha D}
    \begin{longtable}{ |c|p{5cm}|c|c|c| }
    % \begin{tabular}{ |c|c|c|c| }
        \hline
        Index & Premenná & \makecell{Apriórna \\ pravdepodobnosť} & \makecell{Aposteriórna \\ pravdepodobnosť} & \(E(\beta_j|y, X)\) \endhead
        \hline
        1 & Intercept & 100 \% & 100.0 \% & -0.21135 \\
        \hline
        2 & Tržby spolu & 6.76 \% & 0.02 \% & \(-5 \times 10^{-12}\) \\
        \hline
        3 & Tržby vrátane tržieb z predaja Dlhodobého majetku a Cenných papierov a podielov & 6.76 \% & \(4 \times 10^{-6}\) \% & \(2 \times 10^{-16}\) \\
        \hline
        4 & Zisk pred zdanením a úrokmi (EBIT) & 6.76 \% & 0.11 \% & \(-1 \times 10^{-9}\) \\
        \hline
        5 & EBITDA & 6.76 \% & 0.45 \% & \(-4 \times 10^{-9}\) \\
        \hline
        6 & Tržby očistené o Zásoby a Aktiváciu & 6.76 \% & \(9 \times 10^{-5}\) \% & \(-4 \times 10^{-15}\) \\
        \hline
        7 & Účtovný cash flow & 6.76 \% & 69.9 \% & \(-8 \times 10^{-7}\) \\
        \hline
        8 & Náklady na predaný tovar a služby (COGS) & 6.76 \% & 0.01 \% & \(-9 \times 10^{-13}\) \\
        \hline
        9 & Hrubá tvorba zdrojov z prevádzkovej činnosti & 6.76 \% & \(3 \times 10^{-4}\) \% & \(6 \times 10^{-14}\) \\
        \hline
        10 & Čistý prevádzkový zisk po zdanení (NOPAT) & 6.76 \% & 29.93 \% & \(-4 \times 10^{-7}\) \\
        \hline
        11 & Čistý cash flow & 6.76 \% & \(5 \times 10^{-6}\) \% & \(-1 \times 10^{-14}\) \\
        \hline
        12 & Čistý dlh & 6.76 \% & 71.36 \% & \(3 \times 10^{-7}\) \\
        \hline
        13 & Hrubá marža & 6.76 \% & \(1 \times 10^{-4}\) \% & \(-3 \times 10^{-8}\) \\
        \hline
        14 & EBITDA marža & 6.76 \% & \(1 \times 10^{-4}\) \% & \(-4 \times 10^{-9}\) \\
        \hline
        15 & Prevádzková marža & 6.76 \% & \(1 \times 10^{-6}\) \% & \(-1 \times 10^{-10}\) \\
        \hline
        16 & Zisková marža & 6.76 \% & \(6 \times 10^{-5}\) \% & \(-6 \times 10^{-9}\) \\
        \hline
        17 & Marža účtovného cash flow & 6.76 \% & \(2 \times 10^{-5}\) \% & \(1 \times 10^{-9}\) \\
        \hline
        18 & Záväzky/EBITDA & 6.76 \% & 0.02 \% & \(-3 \times 10^{-7}\) \\
        \hline
        19 & Celková zadlženosť & 6.76 \% & 7.77 \% & 0.05955 \\
        \hline
        20 & Stupeň samofinancovania & 6.76 \% & 92.22 \% & -0.72413 \\
        \hline
        21 & Krátkodobá zadlženosť & 6.76 \% & 0.01 \% & 0.00008 \\
        \hline
        22 & Finančná páka & 6.76 \% & \(7 \times 10^{-5}\) \% & \(-8 \times 10^{-10}\) \\
        \hline
        23 & Platobná neschopnosť celková & 6.76 \% & \(6 \times 10^{-4}\) \% & \(3 \times 10^{-9}\) \\
        \hline
        24 & Likvidita 1. stupňa & 6.76 \% & \(2 \times 10^{-5}\) \% & \(-1 \times 10^{-8}\) \\
        \hline
        25 & Likvidita 2. stupňa & 6.76 \% & \(2 \times 10^{-4}\) \% & \(2 \times 10^{-6}\) \\
        \hline
        26 & Likvidita 3. stupňa & 6.76 \% & 0.45 \% & -0.00019 \\
        \hline
        27 & Finančné účty/Aktíva & 6.76 \% & 100.0 \% & -2.73484 \\
        \hline
        28 & Návratnosť aktív & 6.76 \% & 26.08 \% & -0.34939 \\
        \hline
        29 & Návratnosť aktív (EBIT) & 6.76 \% & 73.92 \% & -1.03328 \\
        \hline
        30 & Návratnosť dlhodobého kapitálu (EBIT) & 6.76 \% & \(1 \times 10^{-5}\) \% & \(-4 \times 10^{-12}\) \\
        \hline
        31 & Doba obratu aktív & 6.76 \% & \(2 \times 10^{-5}\) \% & \(5 \times 10^{-13}\) \\
        \hline
        32 & Obrat aktív & 6.76 \% & \(6 \times 10^{-4}\) \% & \(4 \times 10^{-7}\) \\
        \hline
        33 & Obrat obežného majetku & 6.76 \% & \(7 \times 10^{-5}\) \% & \(9 \times 10^{-9}\) \\
        \hline
        34 & Doba splácania záväzkov & 6.76 \% & 1.39 \% & \(2 \times 10^{-8}\) \\
        \hline
        35 & Doba splácania záväzkov vo vzťahu k tržbám & 6.76 \% & 0.02 \% & \(10 \times 10^{-10}\) \\
        \hline
        36 & Doba splácania záväzkov z obchodného styku & 6.76 \% & 99.05 \% & 0.00006 \\
        \hline
        37 & Prirážka & 6.76 \% & \(1 \times 10^{-3}\) \% & \(-1 \times 10^{-7}\) \\
        \hline
        38 & Finančná efektívnosť tržieb & 6.76 \% & \(4 \times 10^{-6}\) \% & \(-2 \times 10^{-9}\) \\
        \hline
        39 & Doba splácania dlhov z čistého cash flow & 6.76 \% & 0.02 \% & \(-7 \times 10^{-9}\) \\
        \hline
        40 & Stupeň oddlženia & 6.76 \% & 0.02 \% & \(3 \times 10^{-7}\) \\
        \hline
        41 & Krátkodobá toková oddlženosť & 6.76 \% & \(3 \times 10^{-5}\) \% & \(3 \times 10^{-10}\) \\
        \hline
        42 & Miera zadlženosti vlastného imania & 6.76 \% & \(1 \times 10^{-3}\) \% & \(-1 \times 10^{-8}\) \\
        \hline
        43 & Altmanovo Z-skóre & 6.76 \% & 0.01 \% & \(-5 \times 10^{-7}\) \\
        \hline
        44 & INDEX 05 & 6.76 \% & 5.44 \% & -0.00013 \\
        \hline
        45 & Binkertov model & 6.76 \% & \(5 \times 10^{-3}\) \% & \(-4 \times 10^{-7}\) \\
        \hline
        46 & In. Bonity & 6.76 \% & \(8 \times 10^{-4}\) \% & \(-1 \times 10^{-8}\) \\
        \hline
        47 & Tafflerov model & 6.76 \% & 0.01 \% & \(3 \times 10^{-6}\) \\
        \hline
        48 & Spolu majetok (zmena v \%) & 6.76 \% & \(1 \times 10^{-4}\) \% & \(-1 \times 10^{-7}\) \\
        \hline
        49 & Obežný majetok (zmena v \%) & 6.76 \% & \(1 \times 10^{-4}\) \% & \(-1 \times 10^{-8}\) \\
        \hline
        50 & Krátkodobé pohľadávky súčet (zmena v \%) & 6.76 \% & \(2 \times 10^{-5}\) \% & \(-3 \times 10^{-10}\) \\
        \hline
        51 & Pohľadávky z obchodného styku (zmena v \%) & 6.76 \% & \(10 \times 10^{-4}\) \% & \(-1 \times 10^{-7}\) \\
        \hline
        52 & Finančné účty súčet (zmena v \%) & 6.76 \% & \(3 \times 10^{-6}\) \% & \(-2 \times 10^{-11}\) \\
        \hline
        53 & Peniaze (zmena v \%) & 6.76 \% & \(7 \times 10^{-6}\) \% & \(-2 \times 10^{-11}\) \\
        \hline
        54 & Účty v bankách (zmena v \%) & 6.76 \% & \(2 \times 10^{-3}\) \% & \(1 \times 10^{-8}\) \\
        \hline
        55 & Spolu vlastné imanie a záväzky (zmena v \%) & 6.76 \% & 0.01 \% & -0.00001 \\
        \hline
        56 & Vlastné imanie (zmena v \%) & 6.76 \% & \(4 \times 10^{-4}\) \% & \(-5 \times 10^{-8}\) \\
        \hline
        57 & Základné imanie súčet (zmena v \%) & 6.76 \% & \(6 \times 10^{-4}\) \% & 0.00001 \\
        \hline
        58 & Výsledok hospodárenia minulých rokov (zmena v \%) & 6.76 \% & \(5 \times 10^{-6}\) \% & \(-1 \times 10^{-11}\) \\
        \hline
        59 & Výsledok hospodárenia za účtovné obdobie po zdanení (zmena v \%) & 6.76 \% & \(2 \times 10^{-7}\) \% & \(-3 \times 10^{-12}\) \\
        \hline
        60 & Záväzky (zmena v \%) & 6.76 \% & \(1 \times 10^{-4}\) \% & \(-1 \times 10^{-8}\) \\
        \hline
        61 & Krátkodobé záväzky súčet (zmena v \%) & 6.76 \% & \(1 \times 10^{-6}\) \% & \(-1 \times 10^{-10}\) \\
        \hline
        62 & Závazky z obchodného styku (zmena v \%) & 6.76 \% & \(2 \times 10^{-3}\) \% & \(-3 \times 10^{-7}\) \\
        \hline
        63 & Daňové závazky a dotácie (zmena v \%) & 6.76 \% & \(6 \times 10^{-4}\) \% & \(-9 \times 10^{-9}\) \\
        \hline
        64 & Výrobná spotreba (zmena v \%) & 6.76 \% & \(4 \times 10^{-5}\) \% & \(2 \times 10^{-8}\) \\
        \hline
        65 & Spotreba materiálu, energie a ostatných neskladovateľný dodávok (zmena v \%) & 6.76 \% & \(4 \times 10^{-3}\) \% & \(-2 \times 10^{-7}\) \\
        \hline
        66 & Služby (zmena v \%) & 6.76 \% & \(1 \times 10^{-4}\) \% & \(-1 \times 10^{-8}\) \\
        \hline
        67 & Pridaná hodnota (zmena v \%) & 6.76 \% & \(8 \times 10^{-7}\) \% & \(-7 \times 10^{-11}\) \\
        \hline
        68 & Výsledok hospodárenia z hospodárskej činnosti (zmena v \%) & 6.76 \% & 0.03 \% & \(-5 \times 10^{-7}\) \\
        \hline
        69 & Ostatné náklady na finančnú činnosť (zmena v \%) & 6.76 \% & \(4 \times 10^{-6}\) \% & \(-4 \times 10^{-10}\) \\
        \hline
        70 & Výsledok hospodárenia z finančnej činnosti (zmena v \%) & 6.76 \% & \(2 \times 10^{-6}\) \% & \(8 \times 10^{-11}\) \\
        \hline
        71 & Výsledok hospodárenia z bežnej činnosti pred zdanením (zmena v \%) & 6.76 \% & \(1 \times 10^{-4}\) \% & \(-4 \times 10^{-9}\) \\
        \hline
        72 & Výsledok hospodárenia z bežnej činnosti po zdanení (zmena v \%) & 6.76 \% & \(5 \times 10^{-6}\) \% & \(-2 \times 10^{-13}\) \\
        \hline
        73 & Výsledok hospodárenia za účtovné obdobie pred zdanením (zmena v \%) & 6.76 \% & \(9 \times 10^{-4}\) \% & \(-4 \times 10^{-8}\) \\
        \hline
        74 & Výsledok hospodárenia za účtovné obdobie po zdanení (zmena v \%) & 6.76 \% & 0.02 \% & \(-4 \times 10^{-7}\) \\
        \hline
        75 & Tržby spolu (zmena v \%) & 6.76 \% & \(2 \times 10^{-7}\) \% & \(-1 \times 10^{-12}\) \\
        \hline
    % \end{tabular}
    \caption{Konečný zoznam všetkých premenných spolu s ich aposteriórnymi pravdepodobnosťami a odhadom \(E(\beta|X, y)\) (iteračná metóda \emph{BACE})}
    \label{bace2_pp}
\end{longtable}

  \newpage
  \section*{Príloha E - Výpočtové aspekty iteračnej metódy \emph{BACE}}
    \label{appendix:e}
    \addcontentsline{toc}{section}{Príloha E}
    \markboth{Príloha E}{Príloha E}
    V tejto prílohe sa bližšie venujeme výpočtovým aspektom iteračnej metódy BACE opísanej v časti\autoref{model_bace_2}.
Pripomeňme si, že pri celkovom počte vysvetľujúcich \(K\) a očakávanom počte premenných v „správnom“ modeli \(\bar{k}\)
sme za apriórne pravdepodobnosti zahrnutia premenných zvolili hodnotu \(\bar{k}/K\), a apriórne pravdepodobnosti modelov sme dopočítali ako:

\[
p_{\text{prior}}(M_i) = \left( \frac{\bar{k}}{K} \right)^{k_i} \left( 1 - \frac{\bar{k}}{K} \right)^{K - k_i},
\]

kde \( k_i \) je počet premenných zahrnutých do modelu \( M_i \).
Po enumerácii nejakého počtu modelov \(c\) spočítavame odhad strednej hodnoty parametra \(\beta\) a aposteriórnu pravdepodobnosť zahrnutia vysvetľujúcej premennej \(X_j\) pomocou vzťahov:

\[
    E(\beta | y) = \sum_{i = 1}^{c} p(M_i | y) \hat{\beta}^{(i)},
\]

\[
    p(\beta_j \neq 0 | y) = \sum_{i = 1}^{c} p(M_i | y) I_{\beta_{j, i} \neq 0},
\]

kde \(\hat{\beta}^{(i)}\) je odhad parametra \(\beta\) pre model \(M_i\), a \(p(M_i | y)\) je aposteriórna pravdepodobnosť modelu \(M_i\), ktorú aproximujeme vzťahom:

\[
    p(M_i | y) = \frac{p(M_i) e^{-\frac{1}{2}BIC(M_i)}}{\sum_{i = 1}^{K} p(M_i) e^{-\frac{1}{2}BIC(M_i)}}.
\]

Pri prístupe, v ktorom sme enumerovali všetky modely s \(\bar{k} = 5\) vysvetľujúcimi premennými, je postup priamočiary –
spočítať odhady \(\beta\) a hodnoty \emph{BIC} pre všetky modely a dopočítať strednú hodnotu \(\beta\) a aposteriórne pravdepodobnosti zahrnutia premenných.
V prípade iteračnej metódy, pri ktorej generujeme nové náhodné modely, až kým hodnota odhadu beta neskonverguje,
treba odhad beta počítať veľakrát počas behu algoritmu, čo robí problém výpočtovo oveľa náročnejším.

Pri implementácii algoritmu sme využili skutočnosť, že odhad beta spočítaný po enumerácii \(I\) modelov vieme využiť pri výpočte odhadu beta po enumerácii \(I + J\) modelov.
Toto tvrdenie ukážeme pre výpočet odhadu \(\beta\) v iteráciách \(I\) a \(I + 1\).
Označme \(\hat{\beta}^{(1:I)}\) odhad parametra \(\beta\) v iterácii \(I\) daný vzťahom:

% \begin{gather*}
%     \hat{\beta}^{(I)} = \sum_{i = 1}^{I} p(M_i | y) \hat{\beta} = \sum_{i = 1}^{I} \frac{p(M_i) e^{-\frac{1}{2}BIC(M_i)}}{\sum_{i = 1}^{K} p(M_i) e^{-\frac{1}{2}BIC(M_i)}} \hat{\beta} = \\
%     \frac{1}{\sum_{i = 1}^{K} p(M_i) e^{-\frac{1}{2}BIC(M_i)}} \sum_{i = 1}^{I} p(M_i) e^{-\frac{1}{2}BIC(M_i)} \hat{\beta}
% \end{gather*}

\[
    \hat{\beta}^{(1:I)} = \sum_{i = 1}^{I} p(M_i | y) \hat{\beta}^{(i)} = \sum_{i = 1}^{I} \frac{p(M_i) e^{-\frac{1}{2}BIC(M_i)}}{\sum_{j = 1}^{I} p(M_j) e^{-\frac{1}{2}BIC(M_j)}} \hat{\beta}^{(i)} =
\]
\begin{equation} \label{appendix__expected_value}
    = \frac{\sum_{i = 1}^{I} p(M_i) e^{-\frac{1}{2}BIC(M_i)} \hat{\beta}^{(i)}}{\sum_{i = 1}^{I} p(M_i) e^{-\frac{1}{2}BIC(M_i)}} = \frac{1}{d_I} \sum_{i = 1}^{I} p(M_i) e^{-\frac{1}{2}BIC(M_i)} \hat{\beta}^{(i)},
\end{equation}

pričom sme zaviedli značenie \(d_n := \sum_{i = 1}^{n} p(M_i) e^{-\frac{1}{2}BIC(M_i)}\).
Pre odhad strednej hodnoty \(\beta\) v iterácii \(I + 1\) (značíme \(\hat{\beta}^{(1:I + 1)}\)) potom platí:

\[
    \hat{\beta}^{(1:I + 1)} = \frac{1}{\sum_{i = 1}^{I + 1} p(M_i) e^{-\frac{1}{2}BIC(M_i)}} \sum_{i = 1}^{I + 1} p(M_i) e^{-\frac{1}{2}BIC(M_i)} \hat{\beta}^{(i)} = 
\]
\[
    = \frac{1}{d_I + p(M_{I + 1}) e^{-\frac{1}{2}BIC(M_{I + 1})}} \left( \left( \sum_{i = 1}^{I} p(M_i) e^{-\frac{1}{2}BIC(M_i)} \hat{\beta}^{(i)} \right) + p(M_{I + 1}) e^{-\frac{1}{2}BIC(M_{I + 1})} \hat{\beta}^{(I+1)} \right) = 
\]
\[
    = \frac{d_I}{d_I + p(M_{I + 1}) e^{-\frac{1}{2}BIC(M_{I + 1})}} \hat{\beta}^{(1:I)} + \frac{p(M_{I + 1}) e^{-\frac{1}{2}BIC(M_{I + 1})} \hat{\beta}^{(I+1)}}{d_I + p(M_{I + 1}) e^{-\frac{1}{2}BIC(M_{I + 1})}}.
\]

Ako vidíme, v iterácii \(I + 1\) nám na výpočet odhadu \(\beta\) stačí odhad \(\hat{\beta}^{(1:I)}\) z predošlej iterácie a hodnota \(d_I\),
ktorá predstavuje súčet apriórnych pravdepodobností modelov prevážených hodnotami \(e^{-\frac{1}{2}BIC(M_{.})}\).
V praxi to znamená, že po výpočte odhadu \(\beta\) v iterácii I môže algoritmus „zahodiť“ apriórne aj aposteriórne pravdepodobnosti enumerovaných modelov,
odhady parametra \(\beta\) pre tieto modely a aj ich hodnoty \emph{BIC}.
Celá informácia potrebná pre ďalší beh algoritmu sa dá zhrnúť len do dvoch hodnôt:
samotného odhadu \(\hat{\beta}^{(1:I)}\) strednej hodnoty parametra \(\beta\) po iterácii \(I\) a hodnoty \(d_I := \sum_{i = 1}^{I} p(M_i) e^{-\frac{1}{2}BIC(M_i)}\).
Napriek tomu, že odhad strednej hodnoty \(\beta\) definujeme pomocou sumy cez všetky enumerované modely (viď vzorec (\ref{posterior_expected_value})),
pri optimálnej implementácii sumu cez všetky modely nebudeme musieť počítať vôbec.

Z uvedeného vyplýva, že algoritmus využívajúci vyššie uvedené vzťahy bude fungovať konštantne rýchlo aj pre obrovský počet enumerovaných modelov (potenciálne desiatky až stovky miliónov).

Podobnú logiku môžeme aplikovať aj na výpočet aposteriórnych pravdepodobností zahrnutia jednotlivých vysvetľujúcich premenných.
Ak pre aposteriórnu pravdepobnosť zahrnutia premennej \(X_j\) do modelu po iterácii \(I\) platí:

\[
    p(\beta_j \neq 0 | y)^{(1:I)} = \sum_{i = 1}^{I} p(M_i | y) I_{\beta_{j, i} \neq 0} = \frac{1}{\sum_{i = 1}^{I} p(M_i) e^{-\frac{1}{2}BIC(M_i)}} \sum_{i = 1}^{I} p(M_i) e^{-\frac{1}{2}BIC(M_i)} I_{\beta_{j, i} \neq 0},
\]

tak po iterácii I + 1 platí:

\[
    p(\beta_j \neq 0 | y)^{(1:I + 1)} =
\]
\[
    = \frac{d_I}{d_I + p(M_{I + 1}) e^{-\frac{1}{2}BIC(M_{I + 1})}} p(\beta_j \neq 0 | y)^{(1:I)} + \frac{p(M_{I + 1}) e^{-\frac{1}{2}BIC(M_{I + 1})} I_{\beta_{j, I + 1} \neq 0}}{d_I + p(M_{I + 1}) e^{-\frac{1}{2}BIC(M_{I + 1})}}.
\]

Opäť vidíme, že algoritmus si v iterácii \(I\) potrebuje zapamätať len dve hodnoty: \(d_I\) a \( p(\beta_j \neq 0 | y)^{(1:I)} \)
(odhad aposteriórnej pravdepodobnosti zahrnutia premennej po \(I\)-tej iterácii).

V praxi nemá význam počítať odhad strednej hodnoty \(\beta\) a aposteriórne pravdepodobnosti zahrnutia premenných v každej iterácii, ale až po enumerácii väčšieho množstva modelov.
V tejto práci sme enumerovali odhad \(\beta\) každých \(10000\) iterácií.
Aj v takomto prípade ale nemusíme zakaždým pri výpočte robiť sumu cez všetky modely.
Ak v iterácii I je odhad \(\beta\) daný vzťahom (\ref{appendix__expected_value}), tak v iterácii \(I + 10000\) môžeme odhad \(\beta\) spočítať ako:

\[
    \hat{\beta}^{(1:I + 10000)} =
\]
\[
    = \frac{d_I}{d_I + \sum_{i = I + 1}^{I + 10000} p(M_i) e^{-\frac{1}{2}BIC(M_i)}} \hat{\beta}^{(1:I)} + \frac{1}{d_I + \sum_{i = I + 1}^{I + 10000} p(M_i) e^{-\frac{1}{2}BIC(M_i)}} \sum_{i = I + 1}^{I + 10000} p(M_i) e^{-\frac{1}{2}BIC(M_i)} \hat{\beta}^{(i)}.
\]

\end{document}
