\thispagestyle{empty}
\section*{Abstrakt}
DRUSKA, Róbert: Predikcia bankrotu firiem v slovenskom prostredí [Diplomová práca],
Univerzita Komenského v Bratislave,
Fakulta matematiky, fyziky a informatiky,
Katedra aplikovanej matematiky a štatistiky,
školiteľ: doc. Mgr. Radoslav Harman, PhD.,
Bratislava, 2022, \textnormal{\pageref*{LastPage}} s.

Práca sa zaoberá problematikou predikcie bankrotu s dôrazom na firmy pôsobiace v slovenskom prostredí.
Hlavným výstupom práce sú tri modely logistickej regresie vytvorené s cieľom hodnotenia finančnej kondície slovenských firiem.
Dôraz bol kladený okrem predikčnej schopnosti aj na interpretovateľnosť jednotlivých modelov.
Pri vytváraní modelov boli použité dve štatistické metódy na voľbu premenných modelu – \emph{lasso} (\emph{least absolute shrinkage and selection operator}) a
\emph{BACE} (\emph{Bayesian averaging of classical estimates}).
Analýza výsledkov naznačuje, že druhá menovaná metóda je vhodná na skúmanie problematík, v ktorých je kladený vysoký dôraz na interpretovateľnosť a pochopiteľnosť výstupov.
Porovnanie modelov v záverečnej časti naznačuje, že modely z práce majú v slovenskom prostredí lepšiu predikčnú schopnosť ako zaužívané modely kreditného rizika,
Altmanovo Z-skóre a index IN05.

\begin{flushleft}
  \textbf{Kľúčové slová:} logistická regresia, predikcia bankrotu, lasso, Bayesian averaging of classical estimates
\end{flushleft}