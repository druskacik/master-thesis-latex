Predikcia bankrotu firiem je predmetom akademického aj profesionálneho výskumu už po dlhé desaťročia.
Motivácia stojaca za hľadaním modelov na predikciu úpadku firiem je rôzna.
Prvé modely vytvorené na základe dát o verejne obchodovateľných spoločnostiach v USA
boli využívané pri analýzach predchádzajúcich nákupom akcií \cite{altman1968}.
Modely kreditného rizika našli veľké využitie v bankovníctve – banky a iné finančné inštitúcie disponujú internými modelmi,
ktorými vyhodnocujú finančnú kondíciu svojich klientov a ich hodnotenia využívajú napr. pri stanovovaní úrokových sadzieb.
Takisto aj súkromné spoločnosti a živnostníci môžu využiť modely kreditného rizika pri bežnom vykonávaní svojej podnikateľskej praxe.

V prvých dvoch kapitolách sa oboznamujeme s problematikou bankrotu na Slovenku a predstavíme si dva známe modely kreditného rizika – Altmanovo Z-skóre a český index IN05.
Tretia kapitola je teoretickou predprípravou k metodike,
ktorú sme v práci použili na modelovanie predikcie bankrotu v slovenskom prostredí.
Oboznámime sa v nej s logistickou regresiou a tiež s metódou \emph{BACE}, ktorá slúži na výber vhodnej kombinácie parametrov do modelu.
V štvrtej kapitole sa bližšie venujeme metodike praktickej časti práce.

V piatej kapitole dôkladne opíšeme proces vytvorenia modelov predikcie bankrotu
od popisu dát cez technické detaily modelovania až po porovnanie nami vytvorených modelov s Altmanovým Z-skóre a indexom IN05
aj medzi sebou navzájom. Posledná, šiesta kapitola obsahuje diskusiu o interpretácii výstupov bankrotových modelov.