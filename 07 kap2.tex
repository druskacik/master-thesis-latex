\section{Známe bankrotové modely}

Riziko úpadku podnikov sa modeluje už desiatky rokov.
Za jeden z prvých formálnych pokusov o štatistickú analýzu úpadku podnikov je považovaná štúdia FitzPatricka z roku 1932 \cite{fitzpatrick},
v ktorej autor pracoval s dátami o 40 firmách (20 v úpadku, 20 zdravých).
Výsledkom FitzPatrickovej štúdie nebol model rizika úpadku, ale analýza jednotlivých finančných ukazovateľov a ich trendov pri prosperujúcich a neprosperujúcich firmách.

Formálnejšie pokusy o modelovanie rizika úpadku firiem začali v 60. rokoch, keď vznikli napr. modely Beavera \cite{beaver}, Tamariho \cite{tamari} alebo Altmana \cite{altman1968},
ktorého Z-skóre vytvorené metódou diskriminačnej analýzy patrí dodnes medzi najpoužívanejšie modely kreditného rizika.
Diskriminačná analýza prevažovala v rámci metód používaných na modelovanie úpadku podnikov až do 80. rokov, keď ju nahradili metódy,
ktoré majú menej matematických požiadaviek na dáta, ako napr. logistická regresia alebo \emph{probit model} \cite{gruszczynski}.

Rozmach moderných technológií, výpočtovej techniky a ľahší prístup k dátam v 21. storočí umožnili použitie mnohých ďalších metód na modelovanie kreditného rizika.
V literatúre vieme nájsť modely neurónovej siete \cite{tsai}, hazardné modely \cite{shumway}, metódu oporných bodov (SVM) \cite{min} atď.

Známe sú aj modely vytvorené pomocou dát o slovenských a českých firmách, napr. bonitné indexy IN vytvorené manželmi Neumaierovými na dátach o českých firmách
či Binkertov model \cite{zalai} slovenského prostredia.
Na tému analýzy kreditného rizika firiem tiež vzniklo niekoľko záverečných prác na FMFI UK \cite{ondrusekova, bohdal}.

\subsection{Altmanovo Z-skóre}

Altmanovo \(Z\)-skóre patrí historicky k najznámejším a najpoužívanejším modelom predikcie bankrotu.
Prvá verzia \(Z\)-skóre bola vytvorená na vzorke \(66\) verejne obchodovateľných firiem, \(33\) prosperujúcich a \(33\) v úpadku \cite{altman1968}.
Išlo o model viacrozmernej diskriminačnej analýzy
\footnote{V anglickej literatúre sa o metóde využitej pri Altmanovom modeli hovorí ako o \emph{multivariate discriminant analysis} (MDA).}
s \(5\) premennými predstavujúcimi finančné údaje o firme.
Vzorku 33 prosperujúcich firiem zvolil Altman tak, že ku každej z \(33\) firiem v úpadku zvolil firmu podobnej veľkosti pôsobiacu v rovnakom odvetví ako daná firma v úpadku.
Pri vzorke firiem v úpadku používal údaje jeden rok pred vyhlásením bankrotu, pri dopárovaných prosperujúcich firmách používal údaje z toho istého roku.

Pôvodná verzia \(Z\)-skóre pracovala s trhovou hodnotou firmy, ktorá je dostupná len pre verejne obchodovateľné spoločnosti. Jej tvar je nasledovný:

\[
    Z = 0.012X_1 + 0.014X_2 + 0.033X_3 + 0.006X_4 + 0.999X_5,
\]

kde

\(X_1 = \) pracovný kapitál / aktíva,

\(X_2 = \) výsledok hospodárenia minulých rokov\footnote{angl. \emph{retained earnings}} / aktíva,
% nerozdelený zisk

\(X_3 = \) EBIT\footnote{zisk pred zdanením a úrokmi (angl. \emph{earnings before interest and taxes})} / aktíva,

\(X_4 = \) trhová hodnota / celkové záväzky,

\(X_5 = \) tržby / aktíva.
\bigskip

Podľa modelu je spoločnosť v prosperujúcej zóne, ak \(Z > 2.99\), a v úpadku, ak \(Z < 1.81\).
Interval \([1.81, 2.99]\) nazývame „šedou“ zónou – ak má firma hodnotu \(Z\) v danom intervale, model firmu jednoznačne nezaraďuje do žiadnej z dvoch skupín
(prosperujúca firma alebo firma v úpadku).

Ako bolo zdôraznené v \cite{altman1983}, pôvodné \(Z\)-skóre bolo vytvorené na vzorke verejne obchodovateľných spoločností a je aplikovateľné len na tento typ firiem.
Pokusom o \emph{ad hoc} modifikáciu modelu (napr. nahradením trhovej hodnoty vlastným imaním) chýbala vedecká exaktnosť.
V roku 1983 vytvoril Altman novú verziu \(Z\)-skóre, aplikovateľnú aj na súkromné spoločnosti, v tvare:

\[
    Z = 0.717X_1 + 0.847X_2 + 3.107X_3 + 0.420X_4 + 0.998X_5,
\]

kde \(X_4 = \) trhová hodnota / celkové záväzky, a ostatné premenné majú rovnaký význam ako v pôvodnom \(Z\)-skóre z roku 1968.

Hranice intervalov pre modifikované \(Z\)-skóre sú:

\( Z > 2.9\) – prosperujúca zóna,

\( Z \in [1.2, 2.9]\) – šedá zóna,

\( Z < 1.2 \) – zóna úpadku.
\bigskip

Keďže väčšina firiem na Slovensku je v súkromnom vlastníctve, v slovenskom prostredí sa využíva hlavne verzia Altmanovho skóre z roku 1983.
Presnosť modelov vytvorených v tejto práci budeme porovnávať práve s touto verziou \(Z\)-skóre.

Aj autor \(Z\)-skóre Edward Altman uznal, že presnosť jeho modelu bola prekonaná inými, konkurenčnými modelmi \cite{altman2017}, napriek tomu však \(Z\)-skóre patrí medzi najpoužívanejšie modely predikcie bankrotu.
Dôvodmi pre túto skutočnosť sú zrejme jednoduchosť a ľahká interpretovateľnosť jeho vzorca a dostatočne vysoká presnosť modelu v globálnom prostredí
(v \cite{altman2017} Altman ukazuje, že presnosť klasifikácie pre väčšinu krajín je vyššia než približne \(0.75\)).


\subsection{Index IN05}

Index IN05 (viď \cite{neumaier1}) je český model slúžiaci na ohodnotenie finančného zdravia spoločnosti.
Model patrí do skupiny bonitných indexov IN vytvorených manželmi Neumaierovými, pričom index IN05 je najnovší z nich a pochádza z roku 2005.
Podobne ako Altmanovo Z-skóre, tento model vznikol na základe diskriminačnej analýzy. Diskriminačná funkcia indexu IN05 má tvar:

\[
    \text{IN05} = 0.13X_1 + 0.04X_2 + 3.97X_3 + 0.21X_4 + 0.09X_5,
\]

kde

\(X_1 = \) aktíva / cudzie zdroje,

\(X_2 = \) EBIT / nákladové úroky,

\(X_3 = \) EBIT / aktíva,

\(X_4 = \) výnosy / aktíva,

\(X_5 = \) obežné aktíva / (krátkodobé závazky + bežné bankové úvery).
\bigskip

Hranice intervalov pre hodnoty indexu IN05 sú:

\( \text{IN05} > 1.6\) – prosperujúca zóna,

\( \text{IN05} \in [0.9, 1.6]\) – šedá zóna,

\( \text{IN05} < 0.9 \) – zóna úpadku.
\bigskip

Podľa publikácie \cite{sav} má pri identifikácii bankrotu index IN05 úspešnosť \(77 \%\), ktorá bola nameraná na vzorke \( 1526 \) českých podnikov.

