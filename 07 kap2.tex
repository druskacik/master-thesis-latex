\section{Známe bankrotné modely}

Riziko úpadku podnikov sa modeluje už desiatky rokov.
Za jeden z prvých formálnych pokusov o štatistickú analýzu úpadku podnikov je považovaná štúdia FitzPatricka z roku 1932,
v ktorej autor pracoval s dátami o 40 firmách (20 v úpadku, 20 zdravých).
Výsledkom FitzPatrickovej štúdie nebol model rizika úpadku, ale analýza jednotlivých finančných ukazovateľov a ich trendov pri prosperujúcich a neprosperujúcich firmách.

Formálnejšie pokusy o modelovanie rizika úpadku firiem začali v 60. rokoch, kedy vznikli napr. modely Beavera, Tamariho, alebo Altmana,
ktorého Z-skóre vytvorené metódou diskriminačnej analýzy patrí dodnes medzi najpoužívanejšie modely kreditného rizika.
Diskriminačná analýza prevažovala vrámci metód používaných na modelovanie úpadku podnikov až do 80. rokov, kedy ju nahradili metódy,
ktoré majú menej matematických požiadaviek na dáta, ako napr. logistická regresia alebo \emph{probit model} \cite{gruszczynski}.

Rozmach moderných technológií, výpočtovej techniky, a ľahší prístup k dátam v 21. storočí umožnili použitie mnohých ďalších metód na modelovanie kreditného rizika.
V literatúre vieme nájsť modely neurónovej siete, hazardné modely, metódu oporného bodu (SVM) atď. (TODO: nájdi viac príkladov aj so zdrojom)

Známe sú aj modely vytvorené pomocou dát o slovenských a českých firmách, napr. bonitné indexy IN vytvorené manželmi Neumaierovými na dátach o českých firmách,
či Binkertov model a model Martina Gulku zo slovenského prostredia. Na tému analýzy kreditného rizika firiem tiež vzniklo niekoľko záverečných prác na FMFI UK \cite{ondrusekova, bohdal}.

\subsection{Altmanovo Z-skóre}

TODO

\subsection{Index IN05}

TODO