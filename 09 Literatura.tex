\begin{thebibliography}{99}

    \bibitem{altman1968} ALTMAN E.: Financial Ratios, Discriminant Analysis and the Prediction of Corporate Bankruptcy. 1968. \emph{Journal of Finance}, 23, 589-609. 

    \bibitem{altman1983} ALTMAN E.: Corporate financial distress: a complete guide to predicting, avoiding, and dealing with bankruptcy. 1983.

    \bibitem{altman2017} ALTMAN E., IWANICZ-DROZDOWSKA M., LAITINEN E., SUVAS A.:
    Financial Distress Prediction in an International Context: A Review and Empirical Analysis of Altman's Z-Score Model. 2017.
    \emph{J Int Financ Manage Account}, 28: 131-171. https://doi.org/10.1111/jifm.12053

    \bibitem{beaver} BEAVER W. H.: Financial Ratios as Predictors of Failure. 1966. \emph{Journal of Accounting Research}, 4, 71-111. https://doi.org/10.2307/2490171

    \bibitem{fitzpatrick} FITZPATRICK P.: A comparison of ratios of successful industrial enterprises with those of failed companies. 1932.
    \emph{The Certifíed Public Accountant}

    \bibitem{tamari} TAMARI M.: Financial ratios as a means of forecasting bankruptcy. 1967.
    \emph{The bankers' magazine : and journal of the money market and railway digest}, 204(1482).

    \bibitem{zalai} ZALAI, Karol. Osobitosti prognózovania finančného vývoja slovenských podnikov. 2000.
    \emph{BIATEC: odborný bankový časopis.} Bratislava: Národná banka Slovenska, 2000, 8(38), 12-14. ISSN 1335-0900.

    \bibitem{sala-i-martin} SALA-I-MARTIN X., DOPPELHOFER G., MILLER R. I.: Determinants of Long-Term Growth: A Bayesian Averaging of Classical Estimates (BACE) Approach. 2004.
    \emph{The American Economic Review} 94(4), 813–35, http://www.jstor.org/stable/3592794

    \bibitem{gruszczynski} GRUSZCZYŃSKI, M.: On Unbalanced Sampling in Bankruptcy Prediction. \emph{International Journal of Financial Studies.} 2019; 7(2):28. https://doi.org/10.3390/ijfs7020028

    \bibitem{ondrusekova} ONDRUŠEKOVÁ, M.: Modely rizika úpadku podnikov. 2018. Diplomová práca. FMFI UK, Bratislava
    
    \bibitem{bohdal} BOHDAL, M.: Analýza bankrotov pomocou metód strojového učenia. 2020. Bakalárska práca. FMFI UK, Bratislava

    \bibitem{protopapadakis} PROTOPAPADAKIS E., NIKLIS D., DOUMPOS M., DOULAMIS A., ZOPOUNIDIS C.: Sample selection algorithms for credit risk modelling through data mining techniques. 2019.
    \emph{International Journal of Data Mining, Modelling and Management.} 11. 103. 10.1504/IJDMMM.2019.098969.

    \bibitem{zmijewski} ZMIJEWSKI M.: Methodological Issues Related to the Estimation of Financial Distress Prediction Models. 1984. \emph{Journal of Accounting Research}, 22, 59–82. https://doi.org/10.2307/2490859

    \bibitem{carlin} CARLIN B. P., LOUIS T. A.: Bayesian Methods for Data Analysis (3rd ed.). 2008. \emph{Chapman and Hall/CRC}. https://doi.org/10.1201/b14884

    \bibitem{tiao} BOX G. E., TIAO G. C.: Bayesian inference in statistical analysis. 1973. \emph{International Statistical Review}, 43, 242.
    
    \bibitem{tsai} TSAI C., Jhen-Wei WU J.: Using neural network ensembles for bankruptcy prediction and credit scoring. 2008.
    \emph{Expert Systems with Applications}. Volume 34, Issue 4. s. 2639-2649, ISSN 0957-4174, https://doi.org/10.1016/j.eswa.2007.05.019.

    \bibitem{shumway} SHUMWAY T.: Forecasting Bankruptcy More Accurately: A Simple Hazard Model. 2001.
    \emph{The Journal of Business}, 74(1), 101–124. https://doi.org/10.1086/209665

    \bibitem{min} MIN J, LEE Y.: Bankruptcy prediction using support vector machine with optimal choice of kernel function parameters. 2005.
    \emph{Expert Systems with Applications.} 28. 603-614. 10.1016/j.eswa.2004.12.008.

    \bibitem{polaci} BIAŁOWOLSKI P., KUSZEWSKI T., WITKOWSKI B.: Macroeconomic Forecasts in Models with Bayesian Averaging of Classical Estimates.
    2012. \emph{Contemporary Economics}. 6. 60. 10.5709/ce.1897-9254.34.

    \bibitem{neumaier1} NEUMAIEROVÁ I., NEUMAIER. I.: Index IN05. 2005.
    \emph{Evropské finanční systémy: Sborník příspěvků z mezinárodní vědecké konference}. 1, 143-148.

    \bibitem{neumaierova} NEUMAIEROVÁ I., NEUMAIER. I.: Proč se ujal index IN a nikoli pyramidový systém ukazatelů INFA. 2009.

    \bibitem{sav} HARUMOVÁ A., JANISOVÁ M.: Hodnotenie slovenských podnikov pomocou skóringovej funkcie. 2014. \emph{Ekonomický časopis}, 62, č. 5, s. 522 - 539.

    \bibitem{boda} BOĎA M., ÚRADNÍČEK V.: The portability of altman’s Z-score model to predicting corporate financial distress of Slovak companies. 2016.
    \emph{Technological and Economic Development of Economy.} 22. 532-553. 10.3846/20294913.2016.1197165.

    \bibitem{begley} BEGLEY J., MING J., WATTS S.: Bankruptcy Classification Errors in the 1980s: An Empirical Analysis of Altman's and Ohlson's Models.
    1996. \emph{Review of Accounting Studies.} 1. 267-284. 10.1007/BF00570833.

    \bibitem{wu} WU Y., GAUNT C., GRAY S.: A Comparison of Alternative Bankruptcy Prediction Models. 
    2010. \emph{Journal of Contemporary Accounting and Economics.} 6. 10.1016/j.jcae.2010.04.002.

    \bibitem{mcculloch} GEORGE E., MCCULLOCH R.: Approaches for Bayesian Variable Selection. 1997. \emph{Statistica Sinica.} 7. 339-373.

    \bibitem{jamespress} JAMES PRESS S.: Subjective and Objective Bayesian Statistics: principles, models, and applications. 2009. \emph{John Wiley \& Sons, Inc.} 10.1002/9780470317105

    \bibitem{glmnet} HASTIE T., JUNYANG Q., KENNETH T.: An Introduction to glmnet. Dostupné na: \url{https://glmnet.stanford.edu/articles/glmnet.html}

    \bibitem{skogsvik} SKOGSVIK K., SKOGSVIK S.: On the choice based sample bias in probabilistic bankruptcy Prediction. 2013.
    \emph{Investment Management and Financial Innovations.} 10. 29-37.

    \bibitem{king} KING G., ZENG L.: Logistic Regression in Rare Events Data. 2001. \emph{Political Analysis.} 9(2). 137-163. doi:10.1093/oxfordjournals.pan.a004868
    
    \bibitem{bodle}  BODLE K. A., CYBINSKI P. J.,MONEM R.: Effect of IFRS adoption on financial reporting quality: Evidence from bankruptcy prediction.
    2016. \emph{Accounting Research Journal, Vol. 29 No. 3.} s. 292-312. https://doi.org/10.1108/ARJ-03-2014-0029

    \bibitem{manski} MANSKI CH. F., LERMAN S. R.: The Estimation of Choice Probabilities from Choice Based Samples. 1977. \emph{Econometrica 45, č. 8} https://doi.org/10.2307/1914121.

    \bibitem{bishop} BISHOP Y., FIENBERG S., HOLLAND P., LIGHT R., MOSTELLER F.: Discrete Multivariate Analysis: Theory and Practice. 1977. \emph{Applied Psychological Measurement.} 1. 10.1177/014662167700100218.

    \bibitem{anderson} ANDERSON J. A.: Separate Sample Logistic Discrimination. 1972. \emph{Biometrika 59, 4. 1}. 19–35. https://doi.org/10.2307/2334611.

    \bibitem{maddala} MADDALA G. S.: Limited-dependent and qualitative variables in econometrics. 1983. \emph{Cambridge [Cambridgeshire]: Cambridge University Press.}

    \bibitem{zbierkazakonov} \emph{Zákon č. 7/2005 Z.z. Zákon o konkurze a reštruktualizácii a o zmene a doplnení niektorých zákonov.}
    Dostupné na: \url{https://www.slov-lex.sk/pravne-predpisy/SK/ZZ/2005/7/20210301}

    \bibitem{finstat} \url{https://finstat.sk/konkurzy-restrukturalizacie}

\end{thebibliography}
