\section{Metodika praktickej časti práce}
\label{metodika}

V praktickej časti diplomovej práce vytvoríme niekoľko modelov logistickej regresie na predikciu bankrotu firiem.
Na modelovanie použijeme finančné dáta o slovenských firmách dostupné z databázy \emph{FinStat}.

Bankrot predstavuje v populácii firiem vzácnu udalosť, preto jedným z problémov bude vybrať vhodnú vzorku na natrénovanie modelov.
Firiem, ktoré za celé obdobie ich existencie neboli v úpadku, je obrovské množstvo, a z tohto množstva použijeme na tréning len malú vzorku – vykonáme tzv. undersampling.
Rôzne metódy undersamplingu v súvislosti s využitím na bankrotné modely sú opísané napr. v \cite{protopapadakis},
zo záverov tejto práce však vyplýva, že komplikovanejšie metódy undersamplingu neprinášajú preukázateľne lepšie výsledky než jednoduchý náhodný výber.

Pre zachovanie jednoduchosti vykonáme undersampling podobným spôsobom, akým sa vykonáva vo väčšine prác venovaných problematike kreditného rizika \cite{zmijewski},
a to dopárovaním jednej prosperujúcej firmy (t.j. firmy, ktorá za celé obdobie svojho pôsobenia nebola v úpadku) ku každej z firiem v úpadku,
pričom dopárovanie vykonáme na základe zhody odvetvia a podobnosti tržieb.

V ďalšej časti sa zameriame na vytvorenie modelu logistickej regresie na predikciu bankrotu firmy.
Počet údajov (parametrov) v databáze \emph{FinStat} dostupných pre každú jednu firmu je vyše 70, nie všetky z týchto parametrov sa ale hodia na predikciu bankrotu.
Na výber najsignifikantnejších ukazovateľov použijeme dve metódy – LASSO (\emph{least absolute shrinkage and selection operator}) a bayesovské priemerovanie modelov.

\subsection{Interpretovateľnosť ako sila logistickej regresie} \label{model interpretability}

Odkedy Edward Altman v roku 1968 uverejnil prvý článok o svojom Z-skóre, bolo na tému predikcie úpadku napísaných desiatky ďalších prác.
Napriek tomu, že mnohé z týchto analýz vyústili do modelov, ktoré mali preukázateľne lepšiu predikčnú schopnosť ako Z-skóre,
Altmanov model diskriminačnej analýzy patrí dodnes medzi najpoužívanejšie a najcitovanejšie modely kreditného rizika [TODO: zdroj?].

Dôvodov pre túto skutočnosť je niekoľko, ale najvýraznejším argumentom pre Altmanovo Z-skóre je jeho relatívna jednoduchosť (pracuje len s 5 údajmi)
a z nej vyplývajúca široká aplikovateľnosť Z-skóre pre mnohé odvetvia či krajiny.
Mnohé z modelov, ktoré preukázali lepšiu predikčnú schopnosť než Z-skóre,
majú omnoho komplikovanejší charakter – pracujú s väčším množstvom parametrov alebo používajú dáta za viac rokov pôsobenia firmy.

Navyše, uvedomme si, že bankrot je silno negatívna udalosť, čo z problematiky jeho predikcie robí výrazne citlivú tému.
V podobných situáciach je často hodnotnejšia predikcia jednoduchším modelom,
ktorého výstup si vie používateľ skontrolovať a zhodnotiť jeho relevanciu pre daný konkrétny vstup, než predikcia komplikovanejším modelom,
ktorý sa síce preukázal ako presnejší na nejakej validačnej vzorke, ale je ťažko interpretovateľný aj pre odborníka z praxe.

Pre účely vytvorenia modelu predikcie úpadku sme zvolili logistickú regresiu práve z toho dôvodu,
že predstavuje kompromis medzi interpretovateľnosťou a predikčnou robustnosťou.
Oproti diskriminačnej analýze, ktorú využil Altman, je model logistickej regresie o niečo komplikovanejší,
na druhej strane ale prináša iné výhody ako napr. zmiernenie matematických požiadaviek na vstupné dáta
či ľahko uchopiteľný obor hôdnot výsledného modelu (interval \([0, 1]\)), a to pri zachovaní jednoduchej interpretovateľnosti jeho parametrov.
Uvedomme si, že logistická regresia nemusí slúžiť len ako nástroj na predikciu a často sa vytvára za účelom štatistickej interpretácie dát.