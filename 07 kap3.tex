\section{Teoretická predpríprava}
\label{teoreticka predpriprava}

\subsection{Logistická regresia}
 
Logistická regresia je štatistická metóda využívaná pri binárnej klasifikácii.
Cieľom logistickej regresie je modelovať pravdepodobnosť nejakej triedy alebo udalosti (vysvetľovanej premennej)
na základe jednej alebo viacerých vysvetľujúcich premenných. TODO

Pred tým, než opíšeme, ako presne logistická regresia funguje, si zadefinujme niekoľko dôležitých pojmov, s ktorými logistická regresia narába.

\begin{defin}
    Logistická funkcia \( \sigma : \mathbb{R} \rightarrow (0, 1) \) je definovaná ako:
    \[
        \sigma(t) = \frac{e^t}{e^t + 1} = \frac{1}{1 + e^{-t}}
    \]
    Inverzná funkcia k logistickej sa nazýva logit funkcia a spĺňa:
    \[
        logit(t) = \sigma^{-1}(t) = \ln\left(\frac{p}{1 - p}\right)
    \]
    pre \( p \in (0, 1) \).
\end{defin}

Na obrázku je zobrazený graf logistickej funkcie na intervale \( (-6, 6) \):

% \begin{center}
%     \begin{tikzpicture}
%     \begin{axis}[
%         xlabel={x},
%         ylabel={y},
%         xmin=-6, xmax=6,
%         ymin=0, ymax=1,
%         xtick={-4,-2,0,2,4},
%         ytick={0,0.2,0.4,0.6,0.8,1},
%         legend pos=north west,
%         ymajorgrids=true,
%         grid style=dashed,
%         height=8cm,
%         width=15cm,
%     ]
    
%     \addplot[
%         color=blue,
%         mark=square,
%         ]
%         coordinates {
%         (2, 1.0)(3, 0.5)(4, 0.25)(5, 0.16667)(6, 0.11111)(7, 0.08333)(8, 0.0625)(9, 0.05)(10, 0.04)(11, 0.03333)(12, 0.02778)(13, 0.02381)(14, 0.02041)(15, 0.01786)(16, 0.015625)
%         };
%         % \legend{Rozptyl}
        
%     \end{axis}
%     \end{tikzpicture}
% \end{center}


\begin{center}
\begin{tikzpicture}[>=stealth]
    \begin{axis}[
        xmin=-6,xmax=6,
        ymin=0,ymax=1,
        axis x line=middle,
        axis y line=middle,
        axis line style=<->,
        xlabel={$x$},
        ylabel={$y$},
        ]
        \addplot[no marks,blue,solid] expression[domain=-6:6,samples=100]{1/(1 + exp(-x))};
    \end{axis}
\end{tikzpicture}
\end{center}

Argumentom prirodzeného logaritmu v logit funkcii je výraz \( \frac{p}{1 - p} \) pre \( p \in (0, 1) \).
Ak takéto \(p\) budeme chápať ako pravdepodobnosť, výraz \( \frac{p}{1 - p} \) predstavuje takzvaný pomer šancí (\emph{angl. odds-ratio}).
Napríklad pri hode kockou je pravdepodobnosť padnutia šestky rovná \( 1/6 \) a pomer šancí je \( \frac{\frac{1}{6}}{1 - \frac{1}{6}} = \frac{\frac{1}{6}}{\frac{5}{6}} = \frac{1}{5} = 1 : 5\),
čo znamená, že 1 možný výsledok hodu kockou zodpovedá šestke a 5 možných výsledkov šestke nezodpovedá.

Nech \( Y \) je binárna vysvetľovaná premenná, ktorej zodpovedá vektor vysvetľujúcich premenných \( x = (x_1, x_2, \ldots, x_k) \).
Pre \( Y \) zjavne platí \( P(Y = 1|x) = 1 - P(Y = 0|x) \).
Logistická regresia predpokladá, že logaritmus pomeru šancí (\emph{angl. log-likelihood ratio}) možno modelovať ako lineárnu funkciu zložiek vektora \( x \).

\[
\ln \left( \frac{P(Y = 1|x)}{P(Y = 0|x)} \right) = \ln \left( \frac{P(Y = 1|x)}{1 - P(Y = 1|x)} \right) = \beta_0 + \beta^T x
\]

Po úpravách sa ľahko dopracujeme k záveru, že pravdepodobnosť \( P(Y = 1|x) \) je rovná výstupu logistickej funkcie, ktorej argument bude hľadaná lineárna kombinácia zložiek vektora \( x \).

\begin{equation} \label{logistic_regression}
P(Y = 1|x) = \frac{1}{1 - e^{-(\beta_0 + \beta^T x})} := h(x)
\end{equation}

Ak logistickú regresiu používame na predikciu (čo budeme robiť neskôr v tejto práci),
vzorec (\ref{logistic_regression}) nám poslúži na výpočet pravdepodobnosti \( P(Y = 1|x) \) pri danom novom \( x \).

\subsubsection{Odhad parametrov v logistickej regresii}

Majme teda zovšebecnený regresný model tvaru

\[
h_\beta(x) = P(Y = 1|x) = \frac{1}{1 - e^{-(\beta_0 + \beta^T x})}
\]

Neznámymi parametrami v tomto modeli sú regresory \( \beta = (\beta_1, \ldots, \beta_k) \).
Na odhad regresných koeficientov sa vo väčšine prípadov používa metóda maximálnej vierohodnosti.
Na rozdiel od obyčajnej lineárnej regresie s normálne rozdelenými chybami, v logistickej regresii nie je možné nájsť exaktné vyjadrenie parametrov \( \beta \),
a na ich odhad sa používa nejaká iteračná metóda.

Keďže \(Y\) nadobúda len hodnoty z \( \{0, 1\} \), pre distribučnú funkciu \(Y\) platí:

\[
P(y | x; \beta ) = h_\beta(x)^y (1 - h_\beta(x))^{1 - y}
\]

Majme namerané vektory dát \( x_1, \ldots, x_n \), \( x_i = x_{i1}, \ldots, x_{ik} \),
a k nim prislúchajúce \( y_1, \ldots, y_n \). Potom pre funkciu vierohodnosti parametra \( \beta \) platí

\[
L(\beta | y; ) TODO
\]

Trénovanie logistickej regresie spočíva v maximalizovaní funkcie vierohodnosti,
čo je ekvivalentné maximalizácii jej logaritmu (\emph{angl. log-likelihood function}), a teda hľadáme

\[
TODO
\]

Na nájdenie maxima log-likelihood funkcie sa používajú iteračné metódy,
napr. funkcia \emph{glm} v základnej verzii jazyka \emph{R} využíva metódu \emph{IRLS} (\emph{iteratively reweighted least squares}).

\subsubsection{Sumárne štatistiky v logistickej regresii}

TODO: Tu napíšem niečo o hodnotách ako \(R^2\) atď., ešte to nemám celkom premyslené.

\subsubsection{Interpretácia parametrov v logistickej regresii}

TODO

\subsection{Bayesian averaging of classical estimates (BACE)}

Dáta, ktoré budeme spracúvať v našej práci, obsahujú 70 vysvetľujúcich premenných pre každý rok pôsobenia firmy.
Prirodzene, nie všetky z nich sa hodia na modelovanie bankrotu.
Naším cieľom bude vytvoriť model, ktorý bude relatívne jednoduchý, a v ktorom budú vystupovať len tie najsignifikantnejšie vysvetľujúce premenné z hľadiska predikcie bankrotu.
V tejto časti si opíšeme jednu z metód, ktorú použijeme, a to \emph{bayesian averaging of classical estimates} (\emph{BACE}).

Základná metóda \emph{BACE} bola vybudovaná za účelom vytvorenia lineárnej regresie, ale jej hlavnú ideu možno ľahko využiť aj pri logistickej regresii.
Názov \emph{bayesian averaging of classical estimates} vznikol na základe toho, že metóda BACE využíva bayesovské priemerovanie modelov (\emph{angl. bayesian averaging}) a klasické odhady parametrov lineárnej regresie metódou najmenších štvorcov.

\subsubsection{Bayesovská štatistika}

Metóda \emph{BACE} narába s teóriou bayesovskej štatistiky.
V klasickej štatistike predpokladáme, že parametre majú pevnú, ale neznámu hodnotu.
Neznáme parametre v klasickej štatistike nie sú náhodnými premennými a nemajú pravdepodobnostnú hustotu.
Narozdiel od toho v bayesovskej štatistike považujeme parametre za náhodné premenné, pričom ich náhodnosť zodpovedá miere neistoty, ktorú o danom parametri máme.

\subsubsection{Bayesovské priemerovanie modelov (bayesian model averaging)}

Jedným zo základných tvrdení bayesovskej štatistiky je tzv. bayesovo pravidlo, ktoré hovorí nasledovne:

\[
P(A|B) = \frac{P(B|A) P(A)}{P(B)}
\]

Bayesovo pravidlo možno zovšeobecniť pre spojité náhodné premenné a ich hustoty. Pre náhodné premenné \(y\) a \( \beta \) platí:

\begin{equation} \label{bayes_rule}
    g(\beta | y) = \frac{f(y | \beta) g(\beta)}{f(y)}
\end{equation}

Aplikujme vzorec (\ref{bayes_rule}) na premenné, ktoré vystupujú v logistickej regresii.
\( \beta \) je vektor parametrov (intercept a koeficienty jednotlivých vysvetľujúcich premenných),
\( g(\beta) \) je jeho priórna hustota, ktorú interpretujeme ako presvedčenie výskumníka o parametri \( \beta \) predtým, než spoznáme dáta.
\( y \) je vektor nameraných dát, \( f(y) \) je jeho hustota.
\( g(\beta | y) \) je posteriórna hustota parametra \( \beta \) (hustota podmienená nameranými dátami \( y \)) a predstavuje presvedčenie výskumníka o parametri \( \beta \) po tom, ako spoznáme dáta \( y \).
Bayesovo pravidlo hovorí o tom, ako skombinovať priórnu informáciu \( g \) s nameranými dátami \( y \) a spočítať naše konečné presvedčenie o parametri \( \beta \) – jeho posteriórnu hustotu \( g(\beta|y) \).

Majme množinu \( X = \{X_1, \ldots X_k\} \) potenciálnych vysvetľujúcich premenných pre model logistickej regresie.
Pri skúmaní signifikantnosti jednotlivých premenných \( X_1, \ldots, X_k \) máme možnosť danú premennú do modelu zaradiť alebo nezaradiť,
existuje teda celkovo \( 2^k \) modelov so všetkými možnými kombináciami parametrov \( X_1, \ldots, X_k \).
(\( M_i \) môže byť reprezentované napr. binárnym vektorom dĺžky \(k\), pričom hodnota na mieste \( j = 1, \ldots, k \) hovorí o zaradení, resp. nezaradení \(j\)-tej vysvetľujúcej premennej do modelu.)

Výskumník zvolí priórne pravdepodobnosti \( P(M_i) \), pričom \( \sum P(M_i) = 1 \) (problematikou voľby priórnych pravdepodobností sa budeme zaoberať neskôr).
Z bayesovho pravidla vyplýva, že:

\[
    g(\beta | y) = \sum_{i = 1}^{2^k} P(M_i) \frac{f(y | \beta) g(\beta | M_i)}{f(y)}
\]

Pre posteriórne pravdepodobnosti vyzerá vzorec nasledovne:

\[
    g(\beta | y) = \sum_{i = 1}^{2^k} P(M_i | y) \frac{f(y | \beta) g(\beta | M_i)}{f(y | M_i)}
\]

Keď poznáme posteriórne pravdepodobnosti (resp. váhy) modelov \( M_i \), umožní nám to výpočet strednej hodnoty parametra \( \beta \).

\[
    E(\beta | y) = \sum_{i = 1}^{2^k} P(M_i | y) \hat{\beta}
\]

kde \( \hat{\beta}_i = E(\beta |y, M_i) \) je odhad parametra \( \beta \) pri použití modelu \( M_i \) metódou maximálnej vierehodnosti
(pozn.: pôvodná metóda BACE pracovala s lineárnou regresiou, kde odhad metódou maximálnej vierohodnosti je vlastne jednoznačný \emph{klasický} odhad metódou najmenších štvorcov;
v prípade logistickej regresie parameter odhadujeme iteračnou metódou).

Posteriórna disperzia parametra beta je daná vzorcom:

\[
    D(\beta | y) = \sum_{i = 1}^{2^k} P(M_i | y) D(\beta | y, M_i) + \sum_{i = 1}^{2^k} P(M_i | y) \left( \hat{\beta}_i - \sum_{i = 1}^{2^k} P(M_i | y) \hat{\beta}_i \right)^2
\]

Metóda BACE navyše umožňuje výpočet posteriórnej pravdepodobnosti zahrnutia premennej do modelu (\emph{angl. posterior inclusion probability}),
čo bude jednoducho súčin posteriórných pravdepodobností modelov obsahujúcich danú vysvetľujúcu premennú.

\[
    P(\beta_j \neq 0 | y) = \sum_{i = 1}^{2^k} P(M_i | y) I_{\beta_{j, i} \neq 0}
\]

kde \( I_{\beta_{j, i} \neq 0} \) je indikátor prítomnosti vysvetľujúcej premennej \( X_j \) v modeli \( M_i \).

Výpočet posteriórnej pravdepodobnosti zahrnutia bude hlavný výstup metódy BACE, ktorý využijeme.
Za signifikantné premenné označíme tie, ktorých posteriórna pravdepodobnosť zahrnutia bude vyššia ako priórna pravdepodobnosť zahrnutia, čo predstavuje štandardný postup pri využívaní tejto metódy pri regresiách.

\subsubsection{Voľba priórnych pravdepodobností}