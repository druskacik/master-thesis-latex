\section{Teoretická predpríprava}
\label{teoreticka predpriprava}

\subsection{Logistická regresia}
 
Logistická regresia je štatistická metóda využívaná pri binárnej klasifikácii.
Cieľom logistickej regresie je modelovať pravdepodobnosť nejakej triedy alebo udalosti (vysvetľovanej premennej)
na základe jednej alebo viacerých vysvetľujúcich premenných. TODO

Pred tým, než opíšeme, ako presne logistická regresia funguje, si zadefinujme niekoľko dôležitých pojmov, s ktorými logistická regresia narába.

\begin{defin}
    Logistická funkcia \( \sigma : \mathbb{R} \rightarrow (0, 1) \) je definovaná ako:
    \[
        \sigma(t) = \frac{e^t}{e^t + 1} = \frac{1}{1 + e^{-t}}
    \]
    Inverzná funkcia k logistickej sa nazýva logit funkcia a spĺňa:
    \[
        logit(t) = \sigma^{-1}(t) = \ln\left(\frac{p}{1 - p}\right)
    \]
    pre \( p \in (0, 1) \).
\end{defin}

Na obrázku je zobrazený graf logistickej funkcie na intervale \( (-6, 6) \):

% \begin{center}
%     \begin{tikzpicture}
%     \begin{axis}[
%         xlabel={x},
%         ylabel={y},
%         xmin=-6, xmax=6,
%         ymin=0, ymax=1,
%         xtick={-4,-2,0,2,4},
%         ytick={0,0.2,0.4,0.6,0.8,1},
%         legend pos=north west,
%         ymajorgrids=true,
%         grid style=dashed,
%         height=8cm,
%         width=15cm,
%     ]
    
%     \addplot[
%         color=blue,
%         mark=square,
%         ]
%         coordinates {
%         (2, 1.0)(3, 0.5)(4, 0.25)(5, 0.16667)(6, 0.11111)(7, 0.08333)(8, 0.0625)(9, 0.05)(10, 0.04)(11, 0.03333)(12, 0.02778)(13, 0.02381)(14, 0.02041)(15, 0.01786)(16, 0.015625)
%         };
%         % \legend{Rozptyl}
        
%     \end{axis}
%     \end{tikzpicture}
% \end{center}


\begin{center}
\begin{tikzpicture}[>=stealth]
    \begin{axis}[
        xmin=-6,xmax=6,
        ymin=0,ymax=1,
        axis x line=middle,
        axis y line=middle,
        axis line style=<->,
        xlabel={$x$},
        ylabel={$y$},
        ]
        \addplot[no marks,blue,solid] expression[domain=-6:6,samples=100]{1/(1 + exp(-x))};
    \end{axis}
\end{tikzpicture}
\end{center}

Argumentom prirodzeného logaritmu v logit funkcii je výraz \( \frac{p}{1 - p} \) pre \( p \in (0, 1) \).
Ak takéto \(p\) budeme chápať ako pravdepodobnosť, výraz \( \frac{p}{1 - p} \) predstavuje takzvaný pomer šancí (\emph{angl. odds-ratio}).
Napríklad pri hode kockou je pravdepodobnosť padnutia šestky rovná \( 1/6 \) a pomer šancí je \( \frac{\frac{1}{6}}{1 - \frac{1}{6}} = \frac{\frac{1}{6}}{\frac{5}{6}} = \frac{1}{5} = 1 : 5\),
čo znamená, že 1 možný výsledok hodu kockou zodpovedá šestke a 5 možných výsledkov šestke nezodpovedá.

Nech \( Y \) je binárna vysvetľovaná premenná, ktorej zodpovedá vektor vysvetľujúcich premenných \( x = (x_1, x_2, \ldots, x_k) \).
Pre \( Y \) zjavne platí \( P(Y = 1|x) = 1 - P(Y = 0|x) \).
Logistická regresia predpokladá, že logaritmus pomeru šancí (\emph{angl. log-likelihood ratio}) možno modelovať ako lineárnu funkciu zložiek vektora \( x \).

\[
\ln \left( \frac{P(Y = 1|x)}{P(Y = 0|x)} \right) = \ln \left( \frac{P(Y = 1|x)}{1 - P(Y = 1|x)} \right) = \beta_0 + \beta^T x
\]

Po úpravách sa ľahko dopracujeme k záveru, že pravdepodobnosť \( P(Y = 1|x) \) je rovná výstupu logistickej funkcie, ktorej argument bude hľadaná lineárna kombinácia zložiek vektora \( x \).

\begin{equation} \label{logistic_regression}
P(Y = 1|x) = \frac{1}{1 - e^{-(\beta_0 + \beta^T x})} := h(x)
\end{equation}

Ak logistickú regresiu používame na predikciu (čo budeme robiť neskôr v tejto práci),
vzorec (\ref{logistic_regression}) nám poslúži na výpočet pravdepodobnosti \( P(Y = 1|x) \) pri danom novom \( x \).

\subsubsection{Odhad parametrov v logistickej regresii}

Majme teda zovšebecnený regresný model tvaru

\[
h_\beta(x) = P(Y = 1|x) = \frac{1}{1 - e^{-(\beta_0 + \beta^T x})}
\]

Neznámymi parametrami v tomto modeli sú regresory \( \beta = (\beta_1, \ldots, \beta_k) \).
Na odhad regresných koeficientov sa vo väčšine prípadov používa metóda maximálnej vierohodnosti.
Na rozdiel od obyčajnej lineárnej regresie s normálne rozdelenými chybami, v logistickej regresii nie je možné nájsť exaktné vyjadrenie parametrov \( \beta \),
a na ich odhad sa používa nejaká iteračná metóda.

Keďže \(Y\) nadobúda len hodnoty z \( \{0, 1\} \), pre distribučnú funkciu \(Y\) platí:

\[
P(y | x; \beta ) = h_\beta(x)^y (1 - h_\beta(x))^{1 - y}
\]

Majme namerané vektory dát \( x_1, \ldots, x_n \), \( x_i = x_{i1}, \ldots, x_{ik} \),
a k nim prislúchajúce \( y_1, \ldots, y_n \). Potom pre funkciu vierohodnosti parametra \( \beta \) platí

\[
L(\beta | y; ) TODO
\]

Trénovanie logistickej regresie spočíva v maximalizovaní funkcie vierohodnosti,
čo je ekvivalentné maximalizácii jej logaritmu (\emph{angl. log-likelihood function}), a teda hľadáme

\[
TODO
\]

Na nájdenie maxima log-likelihood funkcie sa používajú iteračné metódy,
napr. funkcia \emph{glm} v základnej verzii jazyka \emph{R} využíva metódu \emph{IRLS} (\emph{iteratively reweighted least squares}).

\subsubsection{Sumárne štatistiky v logistickej regresii}

TODO: Tu napíšem niečo o hodnotách ako \(R^2\) atď., ešte to nemám celkom premyslené.

\subsubsection{Interpretácia parametrov v logistickej regresii}