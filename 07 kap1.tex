\section{Bankrot na Slovensku}
\label{bankruptcy}

Cieľom tejto práce je vytvoriť model na predikciu úpadku firiem na základe jej finančných údajov.
Na účely modelovania si na začiatok zadefinujme, čo budeme rozumieť pod pojmom úpadok.
Problematikou bankrotu právnických a fyzických osôb na Slovensku sa venuje najmä zákon 7/2005 Z. z. o konkurze a reštrukuturalizácii \cite{zbierkazakonov},
ktorý definuje úpadok nasledovne.

\bigskip
\textit{(1)
Dlžník je v úpadku, ak je platobne neschopný alebo predlžený. Ak dlžník podá návrh na vyhlásenie konkurzu, predpokladá sa, že je v úpadku.}

\textit{(2)
Právnická osoba je platobne neschopná, ak nie je schopná plniť 30 dní po lehote splatnosti aspoň dva peňažné záväzky viac ako jednému veriteľovi. […]}

\textit{(3)
Predlžený je ten, kto je povinný viesť účtovníctvo podľa osobitného predpisu, má viac ako jedného veriteľa a hodnota jeho záväzkov presahuje hodnotu jeho majetku. […]}
\bigskip

Keď je právnická osoba v úpadku, je potrebné vyhlásiť konkurz alebo reštrukturalizáciu.
Konkurzné a reštrukturalizačné konania schvaľuje súd, vďaka čomu poznáme presný dátum ich začiatku, čo sa nám zíde pri modelovaní.

Konkurz znamená speňaženie majetkovej podstaty dlžníka a pomerné uspokojenie jeho veriteľov z tohto majetku.
Konkurz má charakter likvidačného konania a jeho výsledkom je zrušenie a zánik podniku.
V bežnej praxi ide o zdĺhavý proces, ktorý môže trvať aj niekoľko rokov.

Na rozdiel od konkurzu reštrukturalizácia nemá likvidačný charakter a je možné vyhlásiť ju už v čase hroziaceho úpadku.
Cieľom reštrukturalizácie je zachovanie podniku alebo jeho časti a postupné uspokojenie veriteľov spôsobom dohodnutým v reštrukturalizačnom pláne.
Uspokojenie pohľadávok býva v praxi rýchlejšie ako pri konkurze.
V prípade, že reštrukturalizácia bude úspešná, právnická osoba môže naďalej pokračovať v podnikateľskej činnosti,
avšak predpokladom jej úspechu je to, že pohľadávky veriteľov sa budú počas ďalšieho fungovania podniku uspokojovať vo vyššej miere ako v prípade konkurzu.

Spoločným znakom konkurzu a reštrukturalizácie je skutočnosť, že dlžník sa nachádza v krízovej ekonomickej situácii.
Na účely tejto práce budeme \emph{bankrotom} rozumieť začiatok konkurzného alebo reštrukturalizačného konania.

\subsection{Bankrot - zriedkavá udalosť}

Úpadok alebo bankrot možno považovať v rámci populácie aktívnych firiem za zriedkavú udalosť.
Databáza \emph{FinStat}, ktorej dáta budeme využívať v tejto práci, eviduje celkovo 6946 prípadov konkurzného alebo reštrukturalizačného konania za obdobie existencie samostatnej Slovenskej republiky.
Ako uvidíme, nie každé z týchto konaní sa hodí na modelovanie, či už pre nedostatok dostupných dát alebo inú príčinu.

Napríklad, z počtu 6946 firiem v úpadku len pre 3730 firiem poznáme presný dátum začiatku konkurzného alebo reštrukturalizačného konania.
Dátum začiatku konania nepoznáme zväčša pre staršie firmy, ktoré zanikli pred rokom 2010, a pre ktoré by sme v mnohých prípadoch aj tak nemali dostatok dát na úspešné zahrnutie do modelu.

V tabuľke uvádzame počet konkurzov a reštrukturalizácií a počet aktívnych firiem od roku 2011 po rok 2020.

\begin{center}
    \begin{tabular}{ |c|c|c|c|p{3cm}|p{3cm}| }
        \hline
        Rok & Konkurzy & Reštrukturalizácie & Spolu & Celkový počet aktívnych firiem & Proporcia firiem v úpadku \\
        \hline
        2011 & 282 & 21 & 303 & 220465 & 0.14 \% \\
        \hline
        2012 & 296 & 36 & 332 & 240905 & 0.14 \% \\
        \hline
        2013 & 318 & 49 & 367 & 268162 & 0.14 \% \\
        \hline
        2014 & 329 & 56 & 385 & 281474 & 0.14 \% \\
        \hline
        2015 & 325 & 49 & 374 & 294289 & 0.13 \%\\
        \hline
        2016 & 222 & 35 & 257 & 312743 & 0.08 \%\\
        \hline
        2017 & 215 & 22 & 237 & 333039 & 0.07 \% \\
        \hline
        2018 & 256 & 6 & 262 & 353196 & 0.07 \% \\
        \hline
        2019 & 254 & 8 & 262 & 374127 & 0.07 \% \\
        % pozn: v roku 2019 mala jedna firma konanie s kategoriou "ine", ide o firmu s ico 36595098 ... po rucnej kontrole som ju zaradil medzi restrukturalizacie
        \hline
        2020 & 185 & 19 & 204 & 393458 & 0.05 \% \\
        \hline
    \end{tabular}
\end{center}
\bigskip

Ako vidíme, počet firiem v úpadku sa každoročne nachádza pod hranicou \(0.2\) \% z aktívnych firiem.
Z uvedeného vyplýva, že pri modelovaní úpadku firiem budeme pracovať so silno nevyváženými dátami (angl. \emph{angl. unbalanced classes}).
Napriek tomu, že bankrot je v populácii firiem zriedkavý, ide o udalosť, ktorej predpovedanie je zaujímavé pre množstvo strán -
napr. pre manažment firiem, majiteľov firemného kapitálu, poskytovateľov pôžičiek, investorov či poisťovateľov.