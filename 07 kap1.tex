\section{Bankrot na Slovensku}
\label{bankruptcy}

Cieľom tejto práce je vytvoriť model na predikciu úpadku firiem na základe jej finančných údajov.
Pre účely modelovania si na začiatok zadefinujme, čo budeme rozumieť pod pojmom úpadok.
Problematikou bankrotu právnických a fyzických osôb na Slovensku sa venuje najmä zákon 7/2005 Z. z. o konkurze a reštrukuturalizácii,
ktorý definuje úpadok nasledovne.

\bigskip
\textit{(1)
Dlžník je v úpadku, ak je platobne neschopný alebo predlžený. Ak dlžník podá návrh na vyhlásenie konkurzu, predpokladá sa, že je v úpadku.}

\textit{(2)
Právnická osoba je platobne neschopná, ak nie je schopná plniť 30 dní po lehote splatnosti aspoň dva peňažné záväzky viac ako jednému veriteľovi. […]}

\textit{(3)
Predlžený je ten, kto je povinný viesť účtovníctvo podľa osobitného predpisu, má viac ako jedného veriteľa a hodnota jeho záväzkov presahuje hodnotu jeho majetku. […]}
\bigskip

Keď je právnická osoba v úpadku, je potrebné vyhlásiť konkurz alebo reštrukturalizáciu.
Konkurzné a reštrukturalizačné konania schvaľuje súd, vďaka čomu poznáme presný dátum ich začiatku, čo sa nám zíde pri modelovaní.

Konkurz znamená speňaženie majetkovej podstaty dlžníka a pomerné uspokojenie jeho veriteľov z tohto majetku.
Konkurz má charakter likvidačného konania a jeho výsledkom je zrušenie a zánik podniku.
V bežnej praxi ide o zdĺhavý proces, ktorý môže trvať aj niekoľko rokov.

Na rozdiel od konkurzu, reštrukturalizácia nemá likvidačný charakter a je možné vyhlásiť ju už v čase hroziaceho úpadku.
Cieľom reštrukturalizácie je zachovanie podniku alebo jeho časti a postupné uspokojenie veriteľov spôsobom dohodnutým v reštrukturalizačnom pláne.
Uspokojenie pohľadávok býva v praxi rýchlejšie ako pri konkurze.
V prípade, že reštrukturalizácia bude úspešná, právnická osoba može naďalej pokračovať v podnikateľskej činnosti,
avšak predpokladom jej úspechu je to, že pohľadávky veriteľov sa budú počas ďalšieho fungovania podniku uspokojovať vo vyššej miere ako v prípade konkurzu.

Spoločným znakom konkurzu a reštrukturalizácie je skutočnosť, že dlžník sa nachádza v krízovej ekonomickej situácii.
Pre účely tejto práce budeme \emph{bankrotom} rozumieť začiatok konkurzného alebo reštrukturalizačného konania.

\subsection{Bankrot - vzácna udalosť}